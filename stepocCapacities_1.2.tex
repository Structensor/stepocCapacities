\documentclass[11pt]{article}

    \usepackage[breakable]{tcolorbox}
    \usepackage{parskip} % Stop auto-indenting (to mimic markdown behaviour)
    
    \usepackage{iftex}
    \ifPDFTeX
    	\usepackage[T1]{fontenc}
    	\usepackage{mathpazo}
    \else
    	\usepackage{fontspec}
    \fi

    % Basic figure setup, for now with no caption control since it's done
    % automatically by Pandoc (which extracts ![](path) syntax from Markdown).
    \usepackage{graphicx}
    % Maintain compatibility with old templates. Remove in nbconvert 6.0
    \let\Oldincludegraphics\includegraphics
    % Ensure that by default, figures have no caption (until we provide a
    % proper Figure object with a Caption API and a way to capture that
    % in the conversion process - todo).
    \usepackage{caption}
    \DeclareCaptionFormat{nocaption}{}
    \captionsetup{format=nocaption,aboveskip=0pt,belowskip=0pt}

    \usepackage{float}
    \floatplacement{figure}{H} % forces figures to be placed at the correct location
    \usepackage{xcolor} % Allow colors to be defined
    \usepackage{enumerate} % Needed for markdown enumerations to work
    \usepackage{geometry} % Used to adjust the document margins
    \usepackage{amsmath} % Equations
    \usepackage{amssymb} % Equations
    \usepackage{textcomp} % defines textquotesingle
    % Hack from http://tex.stackexchange.com/a/47451/13684:
    \AtBeginDocument{%
        \def\PYZsq{\textquotesingle}% Upright quotes in Pygmentized code
    }
    \usepackage{upquote} % Upright quotes for verbatim code
    \usepackage{eurosym} % defines \euro
    \usepackage[mathletters]{ucs} % Extended unicode (utf-8) support
    \usepackage{fancyvrb} % verbatim replacement that allows latex
    \usepackage{grffile} % extends the file name processing of package graphics 
                         % to support a larger range
    \makeatletter % fix for old versions of grffile with XeLaTeX
    \@ifpackagelater{grffile}{2019/11/01}
    {
      % Do nothing on new versions
    }
    {
      \def\Gread@@xetex#1{%
        \IfFileExists{"\Gin@base".bb}%
        {\Gread@eps{\Gin@base.bb}}%
        {\Gread@@xetex@aux#1}%
      }
    }
    \makeatother
    \usepackage[Export]{adjustbox} % Used to constrain images to a maximum size
    \adjustboxset{max size={0.9\linewidth}{0.9\paperheight}}

    % The hyperref package gives us a pdf with properly built
    % internal navigation ('pdf bookmarks' for the table of contents,
    % internal cross-reference links, web links for URLs, etc.)
    \usepackage{hyperref}
    % The default LaTeX title has an obnoxious amount of whitespace. By default,
    % titling removes some of it. It also provides customization options.
    \usepackage{titling}
    \usepackage{longtable} % longtable support required by pandoc >1.10
    \usepackage{booktabs}  % table support for pandoc > 1.12.2
    \usepackage[inline]{enumitem} % IRkernel/repr support (it uses the enumerate* environment)
    \usepackage[normalem]{ulem} % ulem is needed to support strikethroughs (\sout)
                                % normalem makes italics be italics, not underlines
    \usepackage{mathrsfs}
    

    
    % Colors for the hyperref package
    \definecolor{urlcolor}{rgb}{0,.145,.698}
    \definecolor{linkcolor}{rgb}{.71,0.21,0.01}
    \definecolor{citecolor}{rgb}{.12,.54,.11}

    % ANSI colors
    \definecolor{ansi-black}{HTML}{3E424D}
    \definecolor{ansi-black-intense}{HTML}{282C36}
    \definecolor{ansi-red}{HTML}{E75C58}
    \definecolor{ansi-red-intense}{HTML}{B22B31}
    \definecolor{ansi-green}{HTML}{00A250}
    \definecolor{ansi-green-intense}{HTML}{007427}
    \definecolor{ansi-yellow}{HTML}{DDB62B}
    \definecolor{ansi-yellow-intense}{HTML}{B27D12}
    \definecolor{ansi-blue}{HTML}{208FFB}
    \definecolor{ansi-blue-intense}{HTML}{0065CA}
    \definecolor{ansi-magenta}{HTML}{D160C4}
    \definecolor{ansi-magenta-intense}{HTML}{A03196}
    \definecolor{ansi-cyan}{HTML}{60C6C8}
    \definecolor{ansi-cyan-intense}{HTML}{258F8F}
    \definecolor{ansi-white}{HTML}{C5C1B4}
    \definecolor{ansi-white-intense}{HTML}{A1A6B2}
    \definecolor{ansi-default-inverse-fg}{HTML}{FFFFFF}
    \definecolor{ansi-default-inverse-bg}{HTML}{000000}

    % common color for the border for error outputs.
    \definecolor{outerrorbackground}{HTML}{FFDFDF}

    % commands and environments needed by pandoc snippets
    % extracted from the output of `pandoc -s`
    \providecommand{\tightlist}{%
      \setlength{\itemsep}{0pt}\setlength{\parskip}{0pt}}
    \DefineVerbatimEnvironment{Highlighting}{Verbatim}{commandchars=\\\{\}}
    % Add ',fontsize=\small' for more characters per line
    \newenvironment{Shaded}{}{}
    \newcommand{\KeywordTok}[1]{\textcolor[rgb]{0.00,0.44,0.13}{\textbf{{#1}}}}
    \newcommand{\DataTypeTok}[1]{\textcolor[rgb]{0.56,0.13,0.00}{{#1}}}
    \newcommand{\DecValTok}[1]{\textcolor[rgb]{0.25,0.63,0.44}{{#1}}}
    \newcommand{\BaseNTok}[1]{\textcolor[rgb]{0.25,0.63,0.44}{{#1}}}
    \newcommand{\FloatTok}[1]{\textcolor[rgb]{0.25,0.63,0.44}{{#1}}}
    \newcommand{\CharTok}[1]{\textcolor[rgb]{0.25,0.44,0.63}{{#1}}}
    \newcommand{\StringTok}[1]{\textcolor[rgb]{0.25,0.44,0.63}{{#1}}}
    \newcommand{\CommentTok}[1]{\textcolor[rgb]{0.38,0.63,0.69}{\textit{{#1}}}}
    \newcommand{\OtherTok}[1]{\textcolor[rgb]{0.00,0.44,0.13}{{#1}}}
    \newcommand{\AlertTok}[1]{\textcolor[rgb]{1.00,0.00,0.00}{\textbf{{#1}}}}
    \newcommand{\FunctionTok}[1]{\textcolor[rgb]{0.02,0.16,0.49}{{#1}}}
    \newcommand{\RegionMarkerTok}[1]{{#1}}
    \newcommand{\ErrorTok}[1]{\textcolor[rgb]{1.00,0.00,0.00}{\textbf{{#1}}}}
    \newcommand{\NormalTok}[1]{{#1}}
    
    % Additional commands for more recent versions of Pandoc
    \newcommand{\ConstantTok}[1]{\textcolor[rgb]{0.53,0.00,0.00}{{#1}}}
    \newcommand{\SpecialCharTok}[1]{\textcolor[rgb]{0.25,0.44,0.63}{{#1}}}
    \newcommand{\VerbatimStringTok}[1]{\textcolor[rgb]{0.25,0.44,0.63}{{#1}}}
    \newcommand{\SpecialStringTok}[1]{\textcolor[rgb]{0.73,0.40,0.53}{{#1}}}
    \newcommand{\ImportTok}[1]{{#1}}
    \newcommand{\DocumentationTok}[1]{\textcolor[rgb]{0.73,0.13,0.13}{\textit{{#1}}}}
    \newcommand{\AnnotationTok}[1]{\textcolor[rgb]{0.38,0.63,0.69}{\textbf{\textit{{#1}}}}}
    \newcommand{\CommentVarTok}[1]{\textcolor[rgb]{0.38,0.63,0.69}{\textbf{\textit{{#1}}}}}
    \newcommand{\VariableTok}[1]{\textcolor[rgb]{0.10,0.09,0.49}{{#1}}}
    \newcommand{\ControlFlowTok}[1]{\textcolor[rgb]{0.00,0.44,0.13}{\textbf{{#1}}}}
    \newcommand{\OperatorTok}[1]{\textcolor[rgb]{0.40,0.40,0.40}{{#1}}}
    \newcommand{\BuiltInTok}[1]{{#1}}
    \newcommand{\ExtensionTok}[1]{{#1}}
    \newcommand{\PreprocessorTok}[1]{\textcolor[rgb]{0.74,0.48,0.00}{{#1}}}
    \newcommand{\AttributeTok}[1]{\textcolor[rgb]{0.49,0.56,0.16}{{#1}}}
    \newcommand{\InformationTok}[1]{\textcolor[rgb]{0.38,0.63,0.69}{\textbf{\textit{{#1}}}}}
    \newcommand{\WarningTok}[1]{\textcolor[rgb]{0.38,0.63,0.69}{\textbf{\textit{{#1}}}}}
    
    
    % Define a nice break command that doesn't care if a line doesn't already
    % exist.
    \def\br{\hspace*{\fill} \\* }
    % Math Jax compatibility definitions
    \def\gt{>}
    \def\lt{<}
    \let\Oldtex\TeX
    \let\Oldlatex\LaTeX
    \renewcommand{\TeX}{\textrm{\Oldtex}}
    \renewcommand{\LaTeX}{\textrm{\Oldlatex}}
    % Document parameters
    % Document title
    \title{stepocCapacities\_1.2}
    
    
    
    
    
% Pygments definitions
\makeatletter
\def\PY@reset{\let\PY@it=\relax \let\PY@bf=\relax%
    \let\PY@ul=\relax \let\PY@tc=\relax%
    \let\PY@bc=\relax \let\PY@ff=\relax}
\def\PY@tok#1{\csname PY@tok@#1\endcsname}
\def\PY@toks#1+{\ifx\relax#1\empty\else%
    \PY@tok{#1}\expandafter\PY@toks\fi}
\def\PY@do#1{\PY@bc{\PY@tc{\PY@ul{%
    \PY@it{\PY@bf{\PY@ff{#1}}}}}}}
\def\PY#1#2{\PY@reset\PY@toks#1+\relax+\PY@do{#2}}

\expandafter\def\csname PY@tok@w\endcsname{\def\PY@tc##1{\textcolor[rgb]{0.73,0.73,0.73}{##1}}}
\expandafter\def\csname PY@tok@c\endcsname{\let\PY@it=\textit\def\PY@tc##1{\textcolor[rgb]{0.25,0.50,0.50}{##1}}}
\expandafter\def\csname PY@tok@cp\endcsname{\def\PY@tc##1{\textcolor[rgb]{0.74,0.48,0.00}{##1}}}
\expandafter\def\csname PY@tok@k\endcsname{\let\PY@bf=\textbf\def\PY@tc##1{\textcolor[rgb]{0.00,0.50,0.00}{##1}}}
\expandafter\def\csname PY@tok@kp\endcsname{\def\PY@tc##1{\textcolor[rgb]{0.00,0.50,0.00}{##1}}}
\expandafter\def\csname PY@tok@kt\endcsname{\def\PY@tc##1{\textcolor[rgb]{0.69,0.00,0.25}{##1}}}
\expandafter\def\csname PY@tok@o\endcsname{\def\PY@tc##1{\textcolor[rgb]{0.40,0.40,0.40}{##1}}}
\expandafter\def\csname PY@tok@ow\endcsname{\let\PY@bf=\textbf\def\PY@tc##1{\textcolor[rgb]{0.67,0.13,1.00}{##1}}}
\expandafter\def\csname PY@tok@nb\endcsname{\def\PY@tc##1{\textcolor[rgb]{0.00,0.50,0.00}{##1}}}
\expandafter\def\csname PY@tok@nf\endcsname{\def\PY@tc##1{\textcolor[rgb]{0.00,0.00,1.00}{##1}}}
\expandafter\def\csname PY@tok@nc\endcsname{\let\PY@bf=\textbf\def\PY@tc##1{\textcolor[rgb]{0.00,0.00,1.00}{##1}}}
\expandafter\def\csname PY@tok@nn\endcsname{\let\PY@bf=\textbf\def\PY@tc##1{\textcolor[rgb]{0.00,0.00,1.00}{##1}}}
\expandafter\def\csname PY@tok@ne\endcsname{\let\PY@bf=\textbf\def\PY@tc##1{\textcolor[rgb]{0.82,0.25,0.23}{##1}}}
\expandafter\def\csname PY@tok@nv\endcsname{\def\PY@tc##1{\textcolor[rgb]{0.10,0.09,0.49}{##1}}}
\expandafter\def\csname PY@tok@no\endcsname{\def\PY@tc##1{\textcolor[rgb]{0.53,0.00,0.00}{##1}}}
\expandafter\def\csname PY@tok@nl\endcsname{\def\PY@tc##1{\textcolor[rgb]{0.63,0.63,0.00}{##1}}}
\expandafter\def\csname PY@tok@ni\endcsname{\let\PY@bf=\textbf\def\PY@tc##1{\textcolor[rgb]{0.60,0.60,0.60}{##1}}}
\expandafter\def\csname PY@tok@na\endcsname{\def\PY@tc##1{\textcolor[rgb]{0.49,0.56,0.16}{##1}}}
\expandafter\def\csname PY@tok@nt\endcsname{\let\PY@bf=\textbf\def\PY@tc##1{\textcolor[rgb]{0.00,0.50,0.00}{##1}}}
\expandafter\def\csname PY@tok@nd\endcsname{\def\PY@tc##1{\textcolor[rgb]{0.67,0.13,1.00}{##1}}}
\expandafter\def\csname PY@tok@s\endcsname{\def\PY@tc##1{\textcolor[rgb]{0.73,0.13,0.13}{##1}}}
\expandafter\def\csname PY@tok@sd\endcsname{\let\PY@it=\textit\def\PY@tc##1{\textcolor[rgb]{0.73,0.13,0.13}{##1}}}
\expandafter\def\csname PY@tok@si\endcsname{\let\PY@bf=\textbf\def\PY@tc##1{\textcolor[rgb]{0.73,0.40,0.53}{##1}}}
\expandafter\def\csname PY@tok@se\endcsname{\let\PY@bf=\textbf\def\PY@tc##1{\textcolor[rgb]{0.73,0.40,0.13}{##1}}}
\expandafter\def\csname PY@tok@sr\endcsname{\def\PY@tc##1{\textcolor[rgb]{0.73,0.40,0.53}{##1}}}
\expandafter\def\csname PY@tok@ss\endcsname{\def\PY@tc##1{\textcolor[rgb]{0.10,0.09,0.49}{##1}}}
\expandafter\def\csname PY@tok@sx\endcsname{\def\PY@tc##1{\textcolor[rgb]{0.00,0.50,0.00}{##1}}}
\expandafter\def\csname PY@tok@m\endcsname{\def\PY@tc##1{\textcolor[rgb]{0.40,0.40,0.40}{##1}}}
\expandafter\def\csname PY@tok@gh\endcsname{\let\PY@bf=\textbf\def\PY@tc##1{\textcolor[rgb]{0.00,0.00,0.50}{##1}}}
\expandafter\def\csname PY@tok@gu\endcsname{\let\PY@bf=\textbf\def\PY@tc##1{\textcolor[rgb]{0.50,0.00,0.50}{##1}}}
\expandafter\def\csname PY@tok@gd\endcsname{\def\PY@tc##1{\textcolor[rgb]{0.63,0.00,0.00}{##1}}}
\expandafter\def\csname PY@tok@gi\endcsname{\def\PY@tc##1{\textcolor[rgb]{0.00,0.63,0.00}{##1}}}
\expandafter\def\csname PY@tok@gr\endcsname{\def\PY@tc##1{\textcolor[rgb]{1.00,0.00,0.00}{##1}}}
\expandafter\def\csname PY@tok@ge\endcsname{\let\PY@it=\textit}
\expandafter\def\csname PY@tok@gs\endcsname{\let\PY@bf=\textbf}
\expandafter\def\csname PY@tok@gp\endcsname{\let\PY@bf=\textbf\def\PY@tc##1{\textcolor[rgb]{0.00,0.00,0.50}{##1}}}
\expandafter\def\csname PY@tok@go\endcsname{\def\PY@tc##1{\textcolor[rgb]{0.53,0.53,0.53}{##1}}}
\expandafter\def\csname PY@tok@gt\endcsname{\def\PY@tc##1{\textcolor[rgb]{0.00,0.27,0.87}{##1}}}
\expandafter\def\csname PY@tok@err\endcsname{\def\PY@bc##1{\setlength{\fboxsep}{0pt}\fcolorbox[rgb]{1.00,0.00,0.00}{1,1,1}{\strut ##1}}}
\expandafter\def\csname PY@tok@kc\endcsname{\let\PY@bf=\textbf\def\PY@tc##1{\textcolor[rgb]{0.00,0.50,0.00}{##1}}}
\expandafter\def\csname PY@tok@kd\endcsname{\let\PY@bf=\textbf\def\PY@tc##1{\textcolor[rgb]{0.00,0.50,0.00}{##1}}}
\expandafter\def\csname PY@tok@kn\endcsname{\let\PY@bf=\textbf\def\PY@tc##1{\textcolor[rgb]{0.00,0.50,0.00}{##1}}}
\expandafter\def\csname PY@tok@kr\endcsname{\let\PY@bf=\textbf\def\PY@tc##1{\textcolor[rgb]{0.00,0.50,0.00}{##1}}}
\expandafter\def\csname PY@tok@bp\endcsname{\def\PY@tc##1{\textcolor[rgb]{0.00,0.50,0.00}{##1}}}
\expandafter\def\csname PY@tok@fm\endcsname{\def\PY@tc##1{\textcolor[rgb]{0.00,0.00,1.00}{##1}}}
\expandafter\def\csname PY@tok@vc\endcsname{\def\PY@tc##1{\textcolor[rgb]{0.10,0.09,0.49}{##1}}}
\expandafter\def\csname PY@tok@vg\endcsname{\def\PY@tc##1{\textcolor[rgb]{0.10,0.09,0.49}{##1}}}
\expandafter\def\csname PY@tok@vi\endcsname{\def\PY@tc##1{\textcolor[rgb]{0.10,0.09,0.49}{##1}}}
\expandafter\def\csname PY@tok@vm\endcsname{\def\PY@tc##1{\textcolor[rgb]{0.10,0.09,0.49}{##1}}}
\expandafter\def\csname PY@tok@sa\endcsname{\def\PY@tc##1{\textcolor[rgb]{0.73,0.13,0.13}{##1}}}
\expandafter\def\csname PY@tok@sb\endcsname{\def\PY@tc##1{\textcolor[rgb]{0.73,0.13,0.13}{##1}}}
\expandafter\def\csname PY@tok@sc\endcsname{\def\PY@tc##1{\textcolor[rgb]{0.73,0.13,0.13}{##1}}}
\expandafter\def\csname PY@tok@dl\endcsname{\def\PY@tc##1{\textcolor[rgb]{0.73,0.13,0.13}{##1}}}
\expandafter\def\csname PY@tok@s2\endcsname{\def\PY@tc##1{\textcolor[rgb]{0.73,0.13,0.13}{##1}}}
\expandafter\def\csname PY@tok@sh\endcsname{\def\PY@tc##1{\textcolor[rgb]{0.73,0.13,0.13}{##1}}}
\expandafter\def\csname PY@tok@s1\endcsname{\def\PY@tc##1{\textcolor[rgb]{0.73,0.13,0.13}{##1}}}
\expandafter\def\csname PY@tok@mb\endcsname{\def\PY@tc##1{\textcolor[rgb]{0.40,0.40,0.40}{##1}}}
\expandafter\def\csname PY@tok@mf\endcsname{\def\PY@tc##1{\textcolor[rgb]{0.40,0.40,0.40}{##1}}}
\expandafter\def\csname PY@tok@mh\endcsname{\def\PY@tc##1{\textcolor[rgb]{0.40,0.40,0.40}{##1}}}
\expandafter\def\csname PY@tok@mi\endcsname{\def\PY@tc##1{\textcolor[rgb]{0.40,0.40,0.40}{##1}}}
\expandafter\def\csname PY@tok@il\endcsname{\def\PY@tc##1{\textcolor[rgb]{0.40,0.40,0.40}{##1}}}
\expandafter\def\csname PY@tok@mo\endcsname{\def\PY@tc##1{\textcolor[rgb]{0.40,0.40,0.40}{##1}}}
\expandafter\def\csname PY@tok@ch\endcsname{\let\PY@it=\textit\def\PY@tc##1{\textcolor[rgb]{0.25,0.50,0.50}{##1}}}
\expandafter\def\csname PY@tok@cm\endcsname{\let\PY@it=\textit\def\PY@tc##1{\textcolor[rgb]{0.25,0.50,0.50}{##1}}}
\expandafter\def\csname PY@tok@cpf\endcsname{\let\PY@it=\textit\def\PY@tc##1{\textcolor[rgb]{0.25,0.50,0.50}{##1}}}
\expandafter\def\csname PY@tok@c1\endcsname{\let\PY@it=\textit\def\PY@tc##1{\textcolor[rgb]{0.25,0.50,0.50}{##1}}}
\expandafter\def\csname PY@tok@cs\endcsname{\let\PY@it=\textit\def\PY@tc##1{\textcolor[rgb]{0.25,0.50,0.50}{##1}}}

\def\PYZbs{\char`\\}
\def\PYZus{\char`\_}
\def\PYZob{\char`\{}
\def\PYZcb{\char`\}}
\def\PYZca{\char`\^}
\def\PYZam{\char`\&}
\def\PYZlt{\char`\<}
\def\PYZgt{\char`\>}
\def\PYZsh{\char`\#}
\def\PYZpc{\char`\%}
\def\PYZdl{\char`\$}
\def\PYZhy{\char`\-}
\def\PYZsq{\char`\'}
\def\PYZdq{\char`\"}
\def\PYZti{\char`\~}
% for compatibility with earlier versions
\def\PYZat{@}
\def\PYZlb{[}
\def\PYZrb{]}
\makeatother


    % For linebreaks inside Verbatim environment from package fancyvrb. 
    \makeatletter
        \newbox\Wrappedcontinuationbox 
        \newbox\Wrappedvisiblespacebox 
        \newcommand*\Wrappedvisiblespace {\textcolor{red}{\textvisiblespace}} 
        \newcommand*\Wrappedcontinuationsymbol {\textcolor{red}{\llap{\tiny$\m@th\hookrightarrow$}}} 
        \newcommand*\Wrappedcontinuationindent {3ex } 
        \newcommand*\Wrappedafterbreak {\kern\Wrappedcontinuationindent\copy\Wrappedcontinuationbox} 
        % Take advantage of the already applied Pygments mark-up to insert 
        % potential linebreaks for TeX processing. 
        %        {, <, #, %, $, ' and ": go to next line. 
        %        _, }, ^, &, >, - and ~: stay at end of broken line. 
        % Use of \textquotesingle for straight quote. 
        \newcommand*\Wrappedbreaksatspecials {% 
            \def\PYGZus{\discretionary{\char`\_}{\Wrappedafterbreak}{\char`\_}}% 
            \def\PYGZob{\discretionary{}{\Wrappedafterbreak\char`\{}{\char`\{}}% 
            \def\PYGZcb{\discretionary{\char`\}}{\Wrappedafterbreak}{\char`\}}}% 
            \def\PYGZca{\discretionary{\char`\^}{\Wrappedafterbreak}{\char`\^}}% 
            \def\PYGZam{\discretionary{\char`\&}{\Wrappedafterbreak}{\char`\&}}% 
            \def\PYGZlt{\discretionary{}{\Wrappedafterbreak\char`\<}{\char`\<}}% 
            \def\PYGZgt{\discretionary{\char`\>}{\Wrappedafterbreak}{\char`\>}}% 
            \def\PYGZsh{\discretionary{}{\Wrappedafterbreak\char`\#}{\char`\#}}% 
            \def\PYGZpc{\discretionary{}{\Wrappedafterbreak\char`\%}{\char`\%}}% 
            \def\PYGZdl{\discretionary{}{\Wrappedafterbreak\char`\$}{\char`\$}}% 
            \def\PYGZhy{\discretionary{\char`\-}{\Wrappedafterbreak}{\char`\-}}% 
            \def\PYGZsq{\discretionary{}{\Wrappedafterbreak\textquotesingle}{\textquotesingle}}% 
            \def\PYGZdq{\discretionary{}{\Wrappedafterbreak\char`\"}{\char`\"}}% 
            \def\PYGZti{\discretionary{\char`\~}{\Wrappedafterbreak}{\char`\~}}% 
        } 
        % Some characters . , ; ? ! / are not pygmentized. 
        % This macro makes them "active" and they will insert potential linebreaks 
        \newcommand*\Wrappedbreaksatpunct {% 
            \lccode`\~`\.\lowercase{\def~}{\discretionary{\hbox{\char`\.}}{\Wrappedafterbreak}{\hbox{\char`\.}}}% 
            \lccode`\~`\,\lowercase{\def~}{\discretionary{\hbox{\char`\,}}{\Wrappedafterbreak}{\hbox{\char`\,}}}% 
            \lccode`\~`\;\lowercase{\def~}{\discretionary{\hbox{\char`\;}}{\Wrappedafterbreak}{\hbox{\char`\;}}}% 
            \lccode`\~`\:\lowercase{\def~}{\discretionary{\hbox{\char`\:}}{\Wrappedafterbreak}{\hbox{\char`\:}}}% 
            \lccode`\~`\?\lowercase{\def~}{\discretionary{\hbox{\char`\?}}{\Wrappedafterbreak}{\hbox{\char`\?}}}% 
            \lccode`\~`\!\lowercase{\def~}{\discretionary{\hbox{\char`\!}}{\Wrappedafterbreak}{\hbox{\char`\!}}}% 
            \lccode`\~`\/\lowercase{\def~}{\discretionary{\hbox{\char`\/}}{\Wrappedafterbreak}{\hbox{\char`\/}}}% 
            \catcode`\.\active
            \catcode`\,\active 
            \catcode`\;\active
            \catcode`\:\active
            \catcode`\?\active
            \catcode`\!\active
            \catcode`\/\active 
            \lccode`\~`\~ 	
        }
    \makeatother

    \let\OriginalVerbatim=\Verbatim
    \makeatletter
    \renewcommand{\Verbatim}[1][1]{%
        %\parskip\z@skip
        \sbox\Wrappedcontinuationbox {\Wrappedcontinuationsymbol}%
        \sbox\Wrappedvisiblespacebox {\FV@SetupFont\Wrappedvisiblespace}%
        \def\FancyVerbFormatLine ##1{\hsize\linewidth
            \vtop{\raggedright\hyphenpenalty\z@\exhyphenpenalty\z@
                \doublehyphendemerits\z@\finalhyphendemerits\z@
                \strut ##1\strut}%
        }%
        % If the linebreak is at a space, the latter will be displayed as visible
        % space at end of first line, and a continuation symbol starts next line.
        % Stretch/shrink are however usually zero for typewriter font.
        \def\FV@Space {%
            \nobreak\hskip\z@ plus\fontdimen3\font minus\fontdimen4\font
            \discretionary{\copy\Wrappedvisiblespacebox}{\Wrappedafterbreak}
            {\kern\fontdimen2\font}%
        }%
        
        % Allow breaks at special characters using \PYG... macros.
        \Wrappedbreaksatspecials
        % Breaks at punctuation characters . , ; ? ! and / need catcode=\active 	
        \OriginalVerbatim[#1,codes*=\Wrappedbreaksatpunct]%
    }
    \makeatother

    % Exact colors from NB
    \definecolor{incolor}{HTML}{303F9F}
    \definecolor{outcolor}{HTML}{D84315}
    \definecolor{cellborder}{HTML}{CFCFCF}
    \definecolor{cellbackground}{HTML}{F7F7F7}
    
    % prompt
    \makeatletter
    \newcommand{\boxspacing}{\kern\kvtcb@left@rule\kern\kvtcb@boxsep}
    \makeatother
    \newcommand{\prompt}[4]{
        {\ttfamily\llap{{\color{#2}[#3]:\hspace{3pt}#4}}\vspace{-\baselineskip}}
    }
    

    
    % Prevent overflowing lines due to hard-to-break entities
    \sloppy 
    % Setup hyperref package
    \hypersetup{
      breaklinks=true,  % so long urls are correctly broken across lines
      colorlinks=true,
      urlcolor=urlcolor,
      linkcolor=linkcolor,
      citecolor=citecolor,
      }
    % Slightly bigger margins than the latex defaults
    
    \geometry{verbose,tmargin=1in,bmargin=1in,lmargin=1in,rmargin=1in}
    
    

\begin{document}
    
    \maketitle
    
    

    
    \hypertarget{introduction}{%
\section{Introduction}\label{introduction}}

\hypertarget{background}{%
\subsection{Background}\label{background}}

Stepoc has traditionally been designed as a reinforced masonry element
in bending. Early compression testing of Stepoc provided results for the
characteristic compressive strength of the masonry (\(f_k\)) of circa 18
MPa. Therefore over decades Stepoc has been designed using this figure
when deriving the maximum moment capacity of the section and the lever
arm (\(z\)) when determining reinforcement quantities below this
maximum.

In March 2020 Lucideon undertook compressive strength testing under
instruction from Anderton Concrete. The report published is referenced
(196661) (QT-57177/1/JB)/Ref. 1. From this report a number of salient
points are to be noted:

\begin{itemize}
\tightlist
\item
  256 Stepoc \(f_k\) = 9.92 MPa
\item
  325 Stepoc \(f_k\) = 9.37 MPa
\item
  Failure was deemed to be at the limit of the machine
\end{itemize}

In testing Stepoc with such procedures it is difficult to get truly
meaningful results due to the following factors:

\begin{itemize}
\tightlist
\item
  The stress distribution between the concrete core and the masonry
  shell is dependent on the stiffness of each
\item
  The stress distribution between the core and the shell is dependent
  upon the clean and even application of load
\item
  Due to the size of the section and the high strength of concrete the
  capacity of the machine is easily reached (2000kN)
\end{itemize}

To truly test to failure it is possible that the applied force over a
meter of wall would need to be in excess of 6000kN.

In the design of Stepoc it has been traditionally assumed that the
section of the wall within the compressive stress block is of a uniform
material. This is a simplistic assumption but one that functioned well
when the reported \(f_k\) was in the region of 18MPa. However, BS EN
1996-1-1 places a limited on the moment capacity of the section with the
following equation.

\begin{align} 
M_{Rd} \leq 0.40 f_d b d^2\label{cap1}\\
\end{align}

It can therefore be seen that a reduction in the published \(f_k\) for
Stepoc results in an artificial (based on historic performance of
structures constructed from Stepoc) limit on the bending capacity of the
section. By way of example for 256mm Stepoc using equation \ref{cap1}

    \begin{tcolorbox}[breakable, size=fbox, boxrule=1pt, pad at break*=1mm,colback=cellbackground, colframe=cellborder]
\prompt{In}{incolor}{18}{\boxspacing}
\begin{Verbatim}[commandchars=\\\{\}]
\PY{n}{fk1} \PY{o}{=} \PY{l+m+mf}{9.92}
\PY{n}{fk2} \PY{o}{=} \PY{l+m+mi}{18}
\PY{n}{d} \PY{o}{=} \PY{l+m+mi}{135}
\PY{n}{g} \PY{o}{=} \PY{l+m+mi}{2}
\PY{n}{m1} \PY{o}{=} \PY{l+m+mf}{0.40} \PY{o}{*} \PY{n}{fk1} \PY{o}{*} \PY{l+m+mi}{1000} \PY{o}{*} \PY{n}{d}\PY{o}{*}\PY{o}{*}\PY{l+m+mi}{2} \PY{o}{*} \PY{l+m+mi}{10}\PY{o}{*}\PY{o}{*}\PY{o}{\PYZhy{}}\PY{l+m+mi}{6} \PY{o}{/} \PY{n}{g}
\PY{n}{m2} \PY{o}{=} \PY{l+m+mf}{0.40} \PY{o}{*} \PY{n}{fk2} \PY{o}{*} \PY{l+m+mi}{1000} \PY{o}{*} \PY{n}{d}\PY{o}{*}\PY{o}{*}\PY{l+m+mi}{2} \PY{o}{*} \PY{l+m+mi}{10}\PY{o}{*}\PY{o}{*}\PY{o}{\PYZhy{}}\PY{l+m+mi}{6} \PY{o}{/} \PY{n}{g}


\PY{n+nb}{print}\PY{p}{(}\PY{l+s+s1}{\PYZsq{}}\PY{l+s+s1}{Old fk value for Stepoc = }\PY{l+s+si}{\PYZob{}:.2f\PYZcb{}}\PY{l+s+s1}{MPa with d = 135mm and a partial factor of }\PY{l+s+se}{\PYZbs{}}
\PY{l+s+s1}{2 gives a moment capacity of }\PY{l+s+si}{\PYZob{}:.2f\PYZcb{}}\PY{l+s+s1}{kNm}\PY{l+s+s1}{\PYZsq{}}\PY{o}{.}\PY{n}{format}\PY{p}{(}\PY{n}{fk2}\PY{p}{,} \PY{n}{m2}\PY{p}{)}\PY{p}{)}
\PY{n+nb}{print}\PY{p}{(}\PY{l+s+s1}{\PYZsq{}}\PY{l+s+s1}{New fk value for Stepoc = }\PY{l+s+si}{\PYZob{}:.2f\PYZcb{}}\PY{l+s+s1}{MPa with d = 135mm and a partial factor of }\PY{l+s+se}{\PYZbs{}}
\PY{l+s+s1}{2 gives a moment capacity of }\PY{l+s+si}{\PYZob{}:.2f\PYZcb{}}\PY{l+s+s1}{kNm}\PY{l+s+s1}{\PYZsq{}}\PY{o}{.}\PY{n}{format}\PY{p}{(}\PY{n}{fk1}\PY{p}{,} \PY{n}{m1}\PY{p}{)}\PY{p}{)}

\PY{n}{reduction} \PY{o}{=} \PY{p}{(}\PY{n}{m2} \PY{o}{\PYZhy{}} \PY{n}{m1}\PY{p}{)}\PY{o}{/}\PY{n}{m2} \PY{o}{*} \PY{l+m+mi}{100}

\PY{n+nb}{print}\PY{p}{(}\PY{l+s+s1}{\PYZsq{}}\PY{l+s+s1}{The reduction in capacity = }\PY{l+s+si}{\PYZob{}:.2f\PYZcb{}}\PY{l+s+s1}{\PYZpc{}}\PY{l+s+s1}{\PYZsq{}}\PY{o}{.}\PY{n}{format}\PY{p}{(}\PY{n}{reduction}\PY{p}{)}\PY{p}{)}
\end{Verbatim}
\end{tcolorbox}

    \begin{Verbatim}[commandchars=\\\{\}]
Old fk value for Stepoc = 18.00MPa with d = 135mm and a partial factor of 2
gives a moment capacity of 65.61kNm
New fk value for Stepoc = 9.92MPa with d = 135mm and a partial factor of 2 gives
a moment capacity of 36.16kNm
The reduction in capacity = 44.89\%
    \end{Verbatim}

    It can be see from the above statements that this is a significant
reduction. Since testing has the difficulties noted above and Anderton
Concrete require their new BBA certificate to be published it has been
decided to investigated in greater detail the capacity of the section
accounting for the difference in capacity between the masonry and the
concrete whilst maintaining the new published results for \(fk\).

The calculations and assessment below explore a more complex approach
drawing from structural theory and the basis of BS EN 1996-1-1.

    \hypertarget{extraction-of-a-basis-from-bs-en-1996-1-1}{%
\subsection{Extraction of a basis from BS EN
1996-1-1}\label{extraction-of-a-basis-from-bs-en-1996-1-1}}

BS EN 1996-1-1 limits provides the following conditions relating to the
moment capacity of the section wich are in addtion to equation
\ref{cap1}:

\begin{align}
M_{Rd} = A_s f_y z\\
z = d ( 1 - 0.50 \frac{A_s f_y}{b d f_d}) \leq 0.95 d
\end{align}

Where \(M_{Rd}\) is the design moment resistance, \(A_s\) is the area of
steel reinforcement in tension, \(f_y\) is the design strength of the
reinforcement, \(b\) is the breadth of the section, \(d\) is the
effecitive depth and \(f_d\) is the design strength of the masonry.

These equations allow the maximum depth of the compressive stress block
(depth to neutral axis) \(x\) permitted in the standard to be calculated
as a function of the depth assuming a rectangular stress block. Note for
the purpose of these calculations \(x\) represents \(\lambda x\) in BS
EN 1996-1-1; it being acknowledge that the different \(x-\lambda x\)
permits the neutral axis to be formed.

\begin{align}
   z = d - \frac{x}{2}
   \end{align}

Comparing equations \ref{cap3} and \ref{z1} it is apparent:

\begin{align}
   x = \frac{A_s f_y}{b f_d}
   \end{align}

Which is expected to attain equilibrium of the forces in the
reinforcement and masonry. Extracting \(A_s f_y\) from \ref{cap2} and
apply to \ref{x1} gives:

\begin{align}
   x = \frac{M_{Rd}}{z b f_d}
   \end{align}

We know from \ref{cap1} that M\_\{Rd\} is limited constrained. Applying
this to \ref{x2} yields:

\begin{align}
x = \frac{0.40 d^2}{z}
\end{align}

Apply \ref{z1} for \(z\) in \ref{x3} gives:

\begin{align}
x(d - \frac{x}{2}) = 0.40d^2
\end{align}

Which forms the quadratic: \begin{align}
x^2 - 2 x d + 0.80 d^2 = 0
\end{align}

This quadratic has the viable solution \(x = d(1-\sqrt{0.20})\). This is
therefore the implied limit of the compressive stress block in BS EN
1996-1-1 (\(\lambda x\)).

    \hypertarget{deriviation-of-capacity-accounting-for-masonry-and-concrete-in-the-compressive-stress-block}{%
\section{Deriviation of capacity accounting for masonry and concrete in
the compressive stress
block}\label{deriviation-of-capacity-accounting-for-masonry-and-concrete-in-the-compressive-stress-block}}

The section shall be analysed with 2 rectangular stress blocks. The
combined depth \(x\) of which shall not exceed \(d(1-\sqrt{0.20})\) for
consistency with BS EN 1996-1-1. Since the outer masonry shell is a
rather thin element the following will be assumed in deriving stress
distribution:

\begin{itemize}
\tightlist
\item
  The strain in the shell is constant across its width effectively
  forming axial compression (plastic distribution in bending).
\item
  For compatibility of strains it will be assumed that the strain at the
  interface of the concrete and masonry is equal and that this forms a
  similar plastic distribution. This is to maintain consistency with
  discussion on BS EN 1996-1-1 above in the theoretic assumption that
  the concrete and the masonry can have identical properties.
\item
  Zero strain exists at a position \(x\) from the compressive face
  forming a quasi plastic neutral axis
\end{itemize}

Let \(x_s\) be the depth stress block in the stepoc outer shell and
\(x_c\) be the depth of the stress block in the concrete core. Let
\(\epsilon_{mu}\) be the strain in the masonry shell at the outer face
and \(\epsilon_c\) be the strain at the core / shell interface. To
convert strain to stress \(E_m, E_c\) are the Youngs modulus of
elasticity for masonry and concrete respectively and \(f_{md}, f_{cd}\)
are the design stresses in the masonry and concrete respectively.

\hypertarget{bending-moment-capacity-of-the-section}{%
\subsection{Bending moment capacity of the
section}\label{bending-moment-capacity-of-the-section}}

We note from the above that: \begin{equation}
x = x_s + x_c
\end{equation}

From the strain assumptions to be compatible with BS EN 1996-1-1

\begin{align}
\epsilon_{mu} = \epsilon_c
\end{align}

Converting \ref{s1} to stress:

\begin{align}
\frac{f_{md}}{E_m} = \frac{f_{cd}}{E_c}
\end{align}

Let \(\alpha\) be the modular ratio \(\frac{E_c}{E_m}\) such that from
\ref{s3}:

\begin{align}
f_{cd} = f_{md} \alpha
\end{align}

    Calculate the maximum bending capacity to limit the overall width of the
compressive stress block to that within BS EN 1996-1-1 for elements with
plain masonry in the compression zone. Let \(z_m\) be the lever arm to
the masonry stress block from the reinforcement and \(z_c\) be that for
the concrete.

The maximum forces in the concrete and masonry are:

\begin{align}
\text{In the masonry: } 
F_m = b x_s f_{md}\\
\text{In the concrete: }
F_c = b (x - x_s)f_{cd}
\end{align}

The lever arms to the masonry and concrete compressive stress blocks
are:

\begin{align}
z_m = d - \frac{x_s}{2}\\
z_c = d - \frac{(x + x_s)}{2}
\end{align}

substitute \ref{s4} into \ref{m2} to develop capacity in terms of
masonry strength

\begin{align}
F_c = b \alpha f_{md} (x-x_s)
\end{align}

Calculate the moment capacity of the section by taking the moment of the
masonry and concrete compressive stress blocks about the plane of the
reinforcement and substituting for x:

\begin{align}
M_{Rd} = b x_s f_{md} (d - \frac{x_s}{2}) + b \alpha f_{md}(x-x_s)(d - \frac{(x + x_s)}{2})
\end{align}

Simplifying \ldots{}

\begin{align}
M_{Rd} = b f_{md} \left(x_s (d - \frac{x_s}{2}) +  \alpha (x-x_s) (d - \frac{(x + x_s)}{2})\right)
\end{align}

Equation \ref{m7} gives the moment capacity of the section for under a
known value of \(x\) accounting for the effects of the masonry outer
shell and the inner concrete core to the assumed neutral axis.

    \hypertarget{comparison-of-values-for-maximum-bending-capacity-of-the-section}{%
\section{Comparison of values for maximum bending capacity of the
section}\label{comparison-of-values-for-maximum-bending-capacity-of-the-section}}

To compare results from the above we can compare the maximum moment
capacity when \(x=d(1-\sqrt{0.20})\) using \ref{m7} to that using
original values of \(f_k\) and \ref{cap1}.

Assume:

d = 135mm, b = 1000mm, \(f_k\) of original Stepoc = 18MPa \& \(f_k\) of
current = 9.92MPa

\(x_s\) = 40mm (width of the masonry shell wall), \(\alpha\) = 3
(\(E_m\) circa 10000MPa \& \(E_c\) circa 30000MPa)

Note in the below calculations \(\gamma_M\) (partial factor on material
strength) has been omitted as it is common to both approaches for the
purpose of comparison.

    \begin{tcolorbox}[breakable, size=fbox, boxrule=1pt, pad at break*=1mm,colback=cellbackground, colframe=cellborder]
\prompt{In}{incolor}{5}{\boxspacing}
\begin{Verbatim}[commandchars=\\\{\}]
\PY{n}{xs} \PY{o}{=} \PY{l+m+mi}{39}
\PY{n}{d} \PY{o}{=} \PY{l+m+mi}{152}
\PY{n}{x} \PY{o}{=} \PY{n}{d} \PY{o}{*} \PY{p}{(}\PY{l+m+mi}{1} \PY{o}{\PYZhy{}} \PY{n}{math}\PY{o}{.}\PY{n}{sqrt}\PY{p}{(}\PY{l+m+mf}{0.20}\PY{p}{)}\PY{p}{)}
\PY{n}{a} \PY{o}{=} \PY{l+m+mi}{3}
\PY{n}{b} \PY{o}{=} \PY{l+m+mi}{1000}
\PY{n}{fmd} \PY{o}{=} \PY{l+m+mf}{9.92}
\PY{n}{fmd\PYZus{}old} \PY{o}{=} \PY{l+m+mi}{18}
\PY{n}{c1} \PY{o}{=} \PY{n}{xs}\PY{o}{*}\PY{p}{(}\PY{n}{d} \PY{o}{\PYZhy{}} \PY{n}{xs}\PY{o}{/}\PY{l+m+mi}{2}\PY{p}{)}
\PY{n}{c2} \PY{o}{=} \PY{n}{a} \PY{o}{*}\PY{p}{(}\PY{n}{x}\PY{o}{\PYZhy{}}\PY{n}{xs}\PY{p}{)} \PY{o}{*} \PY{p}{(}\PY{n}{d} \PY{o}{\PYZhy{}} \PY{p}{(}\PY{n}{x}\PY{o}{+}\PY{n}{xs}\PY{p}{)}\PY{o}{/}\PY{l+m+mi}{2}\PY{p}{)}
\PY{n}{c3} \PY{o}{=} \PY{n}{c1} \PY{o}{+} \PY{n}{c2}
\PY{n}{MRd} \PY{o}{=} \PY{n}{c3}\PY{o}{*}\PY{n}{b}\PY{o}{*}\PY{n}{fmd} \PY{o}{*}\PY{l+m+mi}{10}\PY{o}{*}\PY{o}{*}\PY{o}{\PYZhy{}}\PY{l+m+mi}{6}
\PY{n+nb}{print}\PY{p}{(}\PY{l+s+s1}{\PYZsq{}}\PY{l+s+s1}{The maximum moment capacity using current test values \PYZam{} new deriviation = }\PY{l+s+si}{\PYZob{}:.2f\PYZcb{}}\PY{l+s+s1}{kNm}\PY{l+s+s1}{\PYZsq{}}\PY{o}{.}\PY{n}{format}\PY{p}{(}\PY{n}{MRd}\PY{p}{)}\PY{p}{)}

\PY{n}{MRd\PYZus{}old} \PY{o}{=} \PY{l+m+mf}{0.40} \PY{o}{*} \PY{n}{b} \PY{o}{*} \PY{n}{d}\PY{o}{*}\PY{o}{*}\PY{l+m+mi}{2} \PY{o}{*} \PY{n}{fmd\PYZus{}old} \PY{o}{*} \PY{l+m+mi}{10}\PY{o}{*}\PY{o}{*}\PY{o}{\PYZhy{}}\PY{l+m+mi}{6}
\PY{n+nb}{print}\PY{p}{(}\PY{l+s+s1}{\PYZsq{}}\PY{l+s+s1}{The maximum moment capacity using old test values and simplified approach = }\PY{l+s+si}{\PYZob{}:.2f\PYZcb{}}\PY{l+s+s1}{kNm}\PY{l+s+s1}{\PYZsq{}}\PY{o}{.}\PY{n}{format}\PY{p}{(}\PY{n}{MRd\PYZus{}old}\PY{p}{)}\PY{p}{)}
\end{Verbatim}
\end{tcolorbox}

    \begin{Verbatim}[commandchars=\\\{\}]
The maximum moment capacity using current test values \& new deriviation =
172.51kNm
The maximum moment capacity using old test values and simplified approach =
166.35kNm
    \end{Verbatim}

    In the condition of maximum value of \(x\) one can use the above to
attain an `effective' \(f_k\) value based on the published Lucideon
results for use in the BS EN 1996-1-1 moment limiting equation:

    \begin{tcolorbox}[breakable, size=fbox, boxrule=1pt, pad at break*=1mm,colback=cellbackground, colframe=cellborder]
\prompt{In}{incolor}{4}{\boxspacing}
\begin{Verbatim}[commandchars=\\\{\}]
\PY{n}{fk\PYZus{}eff} \PY{o}{=} \PY{n}{MRd} \PY{o}{/} \PY{n}{MRd\PYZus{}old} \PY{o}{*} \PY{n}{fmd\PYZus{}old}
\PY{n+nb}{print}\PY{p}{(}\PY{l+s+s1}{\PYZsq{}}\PY{l+s+s1}{Effective fk in maximum x condition for use in BS EN limiting equation = }\PY{l+s+si}{\PYZob{}:.2f\PYZcb{}}\PY{l+s+s1}{MPa}\PY{l+s+s1}{\PYZsq{}}\PY{o}{.}\PY{n}{format}\PY{p}{(}\PY{n}{fk\PYZus{}eff}\PY{p}{)}\PY{p}{)}
\end{Verbatim}
\end{tcolorbox}

    \begin{Verbatim}[commandchars=\\\{\}]
Effective fk in maximum x condition for use in BS EN limiting equation =
17.24MPa
    \end{Verbatim}

    This result is very similar to the original value and demonstrates that,
with modification to the theory behind the derivation of bending
capacity of the section to account for both concrete and masonry, the
original stepoc capacities can be justified with the new test results.

    \hypertarget{conclusions}{%
\section{Conclusions}\label{conclusions}}

The above calculations indicate that the original maximum bending
capacity of Stepoc can be justified based on current results with
modified theory that deviates from (but is extrapolated), BS EN
1996-1-1.

The results of such findings is that if the current Lucideon test
results are adopted, it should be possible to justify historic designs
in Stepoc using a more complex theoretical approach.

The disadvantage for future designs is that such an approach requires
engineers to assess the capacity of the section from first principles or
adopt the lower \(f_k\) values in the conservative equations of BS EN
1996-1-1 (in relation to Stepoc). In adopting these lower values in the
conservative equations this report deomonstrates that they impose
artificially low limits on the capacity of the Stepoc system.

\hypertarget{recommendations}{%
\section{Recommendations}\label{recommendations}}

It should be possible to develop design charts adopting the above
developed equations (and considering force equilibrium between the
compressive stress blocks and the tension in the reinforcement) that
will mask the complexity of the calculation from design engineers. This
should be investigated.

If possible, Section 6 of the draft BBA certificate should be amended to
permit other methods of calculation of the moment capacity of the
section outside of those listed in BS EN 1996-1-1.


    % Add a bibliography block to the postdoc
    
    
    
\end{document}
