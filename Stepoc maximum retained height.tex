\documentclass[11pt]{article}

    \usepackage[breakable]{tcolorbox}
    \usepackage{parskip} % Stop auto-indenting (to mimic markdown behaviour)
    
    \usepackage{iftex}
    \ifPDFTeX
    	\usepackage[T1]{fontenc}
    	\usepackage{mathpazo}
    \else
    	\usepackage{fontspec}
    \fi

    % Basic figure setup, for now with no caption control since it's done
    % automatically by Pandoc (which extracts ![](path) syntax from Markdown).
    \usepackage{graphicx}
    % Maintain compatibility with old templates. Remove in nbconvert 6.0
    \let\Oldincludegraphics\includegraphics
    % Ensure that by default, figures have no caption (until we provide a
    % proper Figure object with a Caption API and a way to capture that
    % in the conversion process - todo).
    \usepackage{caption}
    \DeclareCaptionFormat{nocaption}{}
    \captionsetup{format=nocaption,aboveskip=0pt,belowskip=0pt}

    \usepackage{float}
    \floatplacement{figure}{H} % forces figures to be placed at the correct location
    \usepackage{xcolor} % Allow colors to be defined
    \usepackage{enumerate} % Needed for markdown enumerations to work
    \usepackage{geometry} % Used to adjust the document margins
    \usepackage{amsmath} % Equations
    \usepackage{amssymb} % Equations
    \usepackage{textcomp} % defines textquotesingle
    % Hack from http://tex.stackexchange.com/a/47451/13684:
    \AtBeginDocument{%
        \def\PYZsq{\textquotesingle}% Upright quotes in Pygmentized code
    }
    \usepackage{upquote} % Upright quotes for verbatim code
    \usepackage{eurosym} % defines \euro
    \usepackage[mathletters]{ucs} % Extended unicode (utf-8) support
    \usepackage{fancyvrb} % verbatim replacement that allows latex
    \usepackage{grffile} % extends the file name processing of package graphics 
                         % to support a larger range
    \makeatletter % fix for old versions of grffile with XeLaTeX
    \@ifpackagelater{grffile}{2019/11/01}
    {
      % Do nothing on new versions
    }
    {
      \def\Gread@@xetex#1{%
        \IfFileExists{"\Gin@base".bb}%
        {\Gread@eps{\Gin@base.bb}}%
        {\Gread@@xetex@aux#1}%
      }
    }
    \makeatother
    \usepackage[Export]{adjustbox} % Used to constrain images to a maximum size
    \adjustboxset{max size={0.9\linewidth}{0.9\paperheight}}

    % The hyperref package gives us a pdf with properly built
    % internal navigation ('pdf bookmarks' for the table of contents,
    % internal cross-reference links, web links for URLs, etc.)
    \usepackage{hyperref}
    % The default LaTeX title has an obnoxious amount of whitespace. By default,
    % titling removes some of it. It also provides customization options.
    \usepackage{titling}
    \usepackage{longtable} % longtable support required by pandoc >1.10
    \usepackage{booktabs}  % table support for pandoc > 1.12.2
    \usepackage[inline]{enumitem} % IRkernel/repr support (it uses the enumerate* environment)
    \usepackage[normalem]{ulem} % ulem is needed to support strikethroughs (\sout)
                                % normalem makes italics be italics, not underlines
    \usepackage{mathrsfs}
    

    
    % Colors for the hyperref package
    \definecolor{urlcolor}{rgb}{0,.145,.698}
    \definecolor{linkcolor}{rgb}{.71,0.21,0.01}
    \definecolor{citecolor}{rgb}{.12,.54,.11}

    % ANSI colors
    \definecolor{ansi-black}{HTML}{3E424D}
    \definecolor{ansi-black-intense}{HTML}{282C36}
    \definecolor{ansi-red}{HTML}{E75C58}
    \definecolor{ansi-red-intense}{HTML}{B22B31}
    \definecolor{ansi-green}{HTML}{00A250}
    \definecolor{ansi-green-intense}{HTML}{007427}
    \definecolor{ansi-yellow}{HTML}{DDB62B}
    \definecolor{ansi-yellow-intense}{HTML}{B27D12}
    \definecolor{ansi-blue}{HTML}{208FFB}
    \definecolor{ansi-blue-intense}{HTML}{0065CA}
    \definecolor{ansi-magenta}{HTML}{D160C4}
    \definecolor{ansi-magenta-intense}{HTML}{A03196}
    \definecolor{ansi-cyan}{HTML}{60C6C8}
    \definecolor{ansi-cyan-intense}{HTML}{258F8F}
    \definecolor{ansi-white}{HTML}{C5C1B4}
    \definecolor{ansi-white-intense}{HTML}{A1A6B2}
    \definecolor{ansi-default-inverse-fg}{HTML}{FFFFFF}
    \definecolor{ansi-default-inverse-bg}{HTML}{000000}

    % common color for the border for error outputs.
    \definecolor{outerrorbackground}{HTML}{FFDFDF}

    % commands and environments needed by pandoc snippets
    % extracted from the output of `pandoc -s`
    \providecommand{\tightlist}{%
      \setlength{\itemsep}{0pt}\setlength{\parskip}{0pt}}
    \DefineVerbatimEnvironment{Highlighting}{Verbatim}{commandchars=\\\{\}}
    % Add ',fontsize=\small' for more characters per line
    \newenvironment{Shaded}{}{}
    \newcommand{\KeywordTok}[1]{\textcolor[rgb]{0.00,0.44,0.13}{\textbf{{#1}}}}
    \newcommand{\DataTypeTok}[1]{\textcolor[rgb]{0.56,0.13,0.00}{{#1}}}
    \newcommand{\DecValTok}[1]{\textcolor[rgb]{0.25,0.63,0.44}{{#1}}}
    \newcommand{\BaseNTok}[1]{\textcolor[rgb]{0.25,0.63,0.44}{{#1}}}
    \newcommand{\FloatTok}[1]{\textcolor[rgb]{0.25,0.63,0.44}{{#1}}}
    \newcommand{\CharTok}[1]{\textcolor[rgb]{0.25,0.44,0.63}{{#1}}}
    \newcommand{\StringTok}[1]{\textcolor[rgb]{0.25,0.44,0.63}{{#1}}}
    \newcommand{\CommentTok}[1]{\textcolor[rgb]{0.38,0.63,0.69}{\textit{{#1}}}}
    \newcommand{\OtherTok}[1]{\textcolor[rgb]{0.00,0.44,0.13}{{#1}}}
    \newcommand{\AlertTok}[1]{\textcolor[rgb]{1.00,0.00,0.00}{\textbf{{#1}}}}
    \newcommand{\FunctionTok}[1]{\textcolor[rgb]{0.02,0.16,0.49}{{#1}}}
    \newcommand{\RegionMarkerTok}[1]{{#1}}
    \newcommand{\ErrorTok}[1]{\textcolor[rgb]{1.00,0.00,0.00}{\textbf{{#1}}}}
    \newcommand{\NormalTok}[1]{{#1}}
    
    % Additional commands for more recent versions of Pandoc
    \newcommand{\ConstantTok}[1]{\textcolor[rgb]{0.53,0.00,0.00}{{#1}}}
    \newcommand{\SpecialCharTok}[1]{\textcolor[rgb]{0.25,0.44,0.63}{{#1}}}
    \newcommand{\VerbatimStringTok}[1]{\textcolor[rgb]{0.25,0.44,0.63}{{#1}}}
    \newcommand{\SpecialStringTok}[1]{\textcolor[rgb]{0.73,0.40,0.53}{{#1}}}
    \newcommand{\ImportTok}[1]{{#1}}
    \newcommand{\DocumentationTok}[1]{\textcolor[rgb]{0.73,0.13,0.13}{\textit{{#1}}}}
    \newcommand{\AnnotationTok}[1]{\textcolor[rgb]{0.38,0.63,0.69}{\textbf{\textit{{#1}}}}}
    \newcommand{\CommentVarTok}[1]{\textcolor[rgb]{0.38,0.63,0.69}{\textbf{\textit{{#1}}}}}
    \newcommand{\VariableTok}[1]{\textcolor[rgb]{0.10,0.09,0.49}{{#1}}}
    \newcommand{\ControlFlowTok}[1]{\textcolor[rgb]{0.00,0.44,0.13}{\textbf{{#1}}}}
    \newcommand{\OperatorTok}[1]{\textcolor[rgb]{0.40,0.40,0.40}{{#1}}}
    \newcommand{\BuiltInTok}[1]{{#1}}
    \newcommand{\ExtensionTok}[1]{{#1}}
    \newcommand{\PreprocessorTok}[1]{\textcolor[rgb]{0.74,0.48,0.00}{{#1}}}
    \newcommand{\AttributeTok}[1]{\textcolor[rgb]{0.49,0.56,0.16}{{#1}}}
    \newcommand{\InformationTok}[1]{\textcolor[rgb]{0.38,0.63,0.69}{\textbf{\textit{{#1}}}}}
    \newcommand{\WarningTok}[1]{\textcolor[rgb]{0.38,0.63,0.69}{\textbf{\textit{{#1}}}}}
    
    
    % Define a nice break command that doesn't care if a line doesn't already
    % exist.
    \def\br{\hspace*{\fill} \\* }
    % Math Jax compatibility definitions
    \def\gt{>}
    \def\lt{<}
    \let\Oldtex\TeX
    \let\Oldlatex\LaTeX
    \renewcommand{\TeX}{\textrm{\Oldtex}}
    \renewcommand{\LaTeX}{\textrm{\Oldlatex}}
    % Document parameters
    % Document title
    \title{Stepoc maximum retained height}
    
    
    
    
    
% Pygments definitions
\makeatletter
\def\PY@reset{\let\PY@it=\relax \let\PY@bf=\relax%
    \let\PY@ul=\relax \let\PY@tc=\relax%
    \let\PY@bc=\relax \let\PY@ff=\relax}
\def\PY@tok#1{\csname PY@tok@#1\endcsname}
\def\PY@toks#1+{\ifx\relax#1\empty\else%
    \PY@tok{#1}\expandafter\PY@toks\fi}
\def\PY@do#1{\PY@bc{\PY@tc{\PY@ul{%
    \PY@it{\PY@bf{\PY@ff{#1}}}}}}}
\def\PY#1#2{\PY@reset\PY@toks#1+\relax+\PY@do{#2}}

\expandafter\def\csname PY@tok@w\endcsname{\def\PY@tc##1{\textcolor[rgb]{0.73,0.73,0.73}{##1}}}
\expandafter\def\csname PY@tok@c\endcsname{\let\PY@it=\textit\def\PY@tc##1{\textcolor[rgb]{0.25,0.50,0.50}{##1}}}
\expandafter\def\csname PY@tok@cp\endcsname{\def\PY@tc##1{\textcolor[rgb]{0.74,0.48,0.00}{##1}}}
\expandafter\def\csname PY@tok@k\endcsname{\let\PY@bf=\textbf\def\PY@tc##1{\textcolor[rgb]{0.00,0.50,0.00}{##1}}}
\expandafter\def\csname PY@tok@kp\endcsname{\def\PY@tc##1{\textcolor[rgb]{0.00,0.50,0.00}{##1}}}
\expandafter\def\csname PY@tok@kt\endcsname{\def\PY@tc##1{\textcolor[rgb]{0.69,0.00,0.25}{##1}}}
\expandafter\def\csname PY@tok@o\endcsname{\def\PY@tc##1{\textcolor[rgb]{0.40,0.40,0.40}{##1}}}
\expandafter\def\csname PY@tok@ow\endcsname{\let\PY@bf=\textbf\def\PY@tc##1{\textcolor[rgb]{0.67,0.13,1.00}{##1}}}
\expandafter\def\csname PY@tok@nb\endcsname{\def\PY@tc##1{\textcolor[rgb]{0.00,0.50,0.00}{##1}}}
\expandafter\def\csname PY@tok@nf\endcsname{\def\PY@tc##1{\textcolor[rgb]{0.00,0.00,1.00}{##1}}}
\expandafter\def\csname PY@tok@nc\endcsname{\let\PY@bf=\textbf\def\PY@tc##1{\textcolor[rgb]{0.00,0.00,1.00}{##1}}}
\expandafter\def\csname PY@tok@nn\endcsname{\let\PY@bf=\textbf\def\PY@tc##1{\textcolor[rgb]{0.00,0.00,1.00}{##1}}}
\expandafter\def\csname PY@tok@ne\endcsname{\let\PY@bf=\textbf\def\PY@tc##1{\textcolor[rgb]{0.82,0.25,0.23}{##1}}}
\expandafter\def\csname PY@tok@nv\endcsname{\def\PY@tc##1{\textcolor[rgb]{0.10,0.09,0.49}{##1}}}
\expandafter\def\csname PY@tok@no\endcsname{\def\PY@tc##1{\textcolor[rgb]{0.53,0.00,0.00}{##1}}}
\expandafter\def\csname PY@tok@nl\endcsname{\def\PY@tc##1{\textcolor[rgb]{0.63,0.63,0.00}{##1}}}
\expandafter\def\csname PY@tok@ni\endcsname{\let\PY@bf=\textbf\def\PY@tc##1{\textcolor[rgb]{0.60,0.60,0.60}{##1}}}
\expandafter\def\csname PY@tok@na\endcsname{\def\PY@tc##1{\textcolor[rgb]{0.49,0.56,0.16}{##1}}}
\expandafter\def\csname PY@tok@nt\endcsname{\let\PY@bf=\textbf\def\PY@tc##1{\textcolor[rgb]{0.00,0.50,0.00}{##1}}}
\expandafter\def\csname PY@tok@nd\endcsname{\def\PY@tc##1{\textcolor[rgb]{0.67,0.13,1.00}{##1}}}
\expandafter\def\csname PY@tok@s\endcsname{\def\PY@tc##1{\textcolor[rgb]{0.73,0.13,0.13}{##1}}}
\expandafter\def\csname PY@tok@sd\endcsname{\let\PY@it=\textit\def\PY@tc##1{\textcolor[rgb]{0.73,0.13,0.13}{##1}}}
\expandafter\def\csname PY@tok@si\endcsname{\let\PY@bf=\textbf\def\PY@tc##1{\textcolor[rgb]{0.73,0.40,0.53}{##1}}}
\expandafter\def\csname PY@tok@se\endcsname{\let\PY@bf=\textbf\def\PY@tc##1{\textcolor[rgb]{0.73,0.40,0.13}{##1}}}
\expandafter\def\csname PY@tok@sr\endcsname{\def\PY@tc##1{\textcolor[rgb]{0.73,0.40,0.53}{##1}}}
\expandafter\def\csname PY@tok@ss\endcsname{\def\PY@tc##1{\textcolor[rgb]{0.10,0.09,0.49}{##1}}}
\expandafter\def\csname PY@tok@sx\endcsname{\def\PY@tc##1{\textcolor[rgb]{0.00,0.50,0.00}{##1}}}
\expandafter\def\csname PY@tok@m\endcsname{\def\PY@tc##1{\textcolor[rgb]{0.40,0.40,0.40}{##1}}}
\expandafter\def\csname PY@tok@gh\endcsname{\let\PY@bf=\textbf\def\PY@tc##1{\textcolor[rgb]{0.00,0.00,0.50}{##1}}}
\expandafter\def\csname PY@tok@gu\endcsname{\let\PY@bf=\textbf\def\PY@tc##1{\textcolor[rgb]{0.50,0.00,0.50}{##1}}}
\expandafter\def\csname PY@tok@gd\endcsname{\def\PY@tc##1{\textcolor[rgb]{0.63,0.00,0.00}{##1}}}
\expandafter\def\csname PY@tok@gi\endcsname{\def\PY@tc##1{\textcolor[rgb]{0.00,0.63,0.00}{##1}}}
\expandafter\def\csname PY@tok@gr\endcsname{\def\PY@tc##1{\textcolor[rgb]{1.00,0.00,0.00}{##1}}}
\expandafter\def\csname PY@tok@ge\endcsname{\let\PY@it=\textit}
\expandafter\def\csname PY@tok@gs\endcsname{\let\PY@bf=\textbf}
\expandafter\def\csname PY@tok@gp\endcsname{\let\PY@bf=\textbf\def\PY@tc##1{\textcolor[rgb]{0.00,0.00,0.50}{##1}}}
\expandafter\def\csname PY@tok@go\endcsname{\def\PY@tc##1{\textcolor[rgb]{0.53,0.53,0.53}{##1}}}
\expandafter\def\csname PY@tok@gt\endcsname{\def\PY@tc##1{\textcolor[rgb]{0.00,0.27,0.87}{##1}}}
\expandafter\def\csname PY@tok@err\endcsname{\def\PY@bc##1{\setlength{\fboxsep}{0pt}\fcolorbox[rgb]{1.00,0.00,0.00}{1,1,1}{\strut ##1}}}
\expandafter\def\csname PY@tok@kc\endcsname{\let\PY@bf=\textbf\def\PY@tc##1{\textcolor[rgb]{0.00,0.50,0.00}{##1}}}
\expandafter\def\csname PY@tok@kd\endcsname{\let\PY@bf=\textbf\def\PY@tc##1{\textcolor[rgb]{0.00,0.50,0.00}{##1}}}
\expandafter\def\csname PY@tok@kn\endcsname{\let\PY@bf=\textbf\def\PY@tc##1{\textcolor[rgb]{0.00,0.50,0.00}{##1}}}
\expandafter\def\csname PY@tok@kr\endcsname{\let\PY@bf=\textbf\def\PY@tc##1{\textcolor[rgb]{0.00,0.50,0.00}{##1}}}
\expandafter\def\csname PY@tok@bp\endcsname{\def\PY@tc##1{\textcolor[rgb]{0.00,0.50,0.00}{##1}}}
\expandafter\def\csname PY@tok@fm\endcsname{\def\PY@tc##1{\textcolor[rgb]{0.00,0.00,1.00}{##1}}}
\expandafter\def\csname PY@tok@vc\endcsname{\def\PY@tc##1{\textcolor[rgb]{0.10,0.09,0.49}{##1}}}
\expandafter\def\csname PY@tok@vg\endcsname{\def\PY@tc##1{\textcolor[rgb]{0.10,0.09,0.49}{##1}}}
\expandafter\def\csname PY@tok@vi\endcsname{\def\PY@tc##1{\textcolor[rgb]{0.10,0.09,0.49}{##1}}}
\expandafter\def\csname PY@tok@vm\endcsname{\def\PY@tc##1{\textcolor[rgb]{0.10,0.09,0.49}{##1}}}
\expandafter\def\csname PY@tok@sa\endcsname{\def\PY@tc##1{\textcolor[rgb]{0.73,0.13,0.13}{##1}}}
\expandafter\def\csname PY@tok@sb\endcsname{\def\PY@tc##1{\textcolor[rgb]{0.73,0.13,0.13}{##1}}}
\expandafter\def\csname PY@tok@sc\endcsname{\def\PY@tc##1{\textcolor[rgb]{0.73,0.13,0.13}{##1}}}
\expandafter\def\csname PY@tok@dl\endcsname{\def\PY@tc##1{\textcolor[rgb]{0.73,0.13,0.13}{##1}}}
\expandafter\def\csname PY@tok@s2\endcsname{\def\PY@tc##1{\textcolor[rgb]{0.73,0.13,0.13}{##1}}}
\expandafter\def\csname PY@tok@sh\endcsname{\def\PY@tc##1{\textcolor[rgb]{0.73,0.13,0.13}{##1}}}
\expandafter\def\csname PY@tok@s1\endcsname{\def\PY@tc##1{\textcolor[rgb]{0.73,0.13,0.13}{##1}}}
\expandafter\def\csname PY@tok@mb\endcsname{\def\PY@tc##1{\textcolor[rgb]{0.40,0.40,0.40}{##1}}}
\expandafter\def\csname PY@tok@mf\endcsname{\def\PY@tc##1{\textcolor[rgb]{0.40,0.40,0.40}{##1}}}
\expandafter\def\csname PY@tok@mh\endcsname{\def\PY@tc##1{\textcolor[rgb]{0.40,0.40,0.40}{##1}}}
\expandafter\def\csname PY@tok@mi\endcsname{\def\PY@tc##1{\textcolor[rgb]{0.40,0.40,0.40}{##1}}}
\expandafter\def\csname PY@tok@il\endcsname{\def\PY@tc##1{\textcolor[rgb]{0.40,0.40,0.40}{##1}}}
\expandafter\def\csname PY@tok@mo\endcsname{\def\PY@tc##1{\textcolor[rgb]{0.40,0.40,0.40}{##1}}}
\expandafter\def\csname PY@tok@ch\endcsname{\let\PY@it=\textit\def\PY@tc##1{\textcolor[rgb]{0.25,0.50,0.50}{##1}}}
\expandafter\def\csname PY@tok@cm\endcsname{\let\PY@it=\textit\def\PY@tc##1{\textcolor[rgb]{0.25,0.50,0.50}{##1}}}
\expandafter\def\csname PY@tok@cpf\endcsname{\let\PY@it=\textit\def\PY@tc##1{\textcolor[rgb]{0.25,0.50,0.50}{##1}}}
\expandafter\def\csname PY@tok@c1\endcsname{\let\PY@it=\textit\def\PY@tc##1{\textcolor[rgb]{0.25,0.50,0.50}{##1}}}
\expandafter\def\csname PY@tok@cs\endcsname{\let\PY@it=\textit\def\PY@tc##1{\textcolor[rgb]{0.25,0.50,0.50}{##1}}}

\def\PYZbs{\char`\\}
\def\PYZus{\char`\_}
\def\PYZob{\char`\{}
\def\PYZcb{\char`\}}
\def\PYZca{\char`\^}
\def\PYZam{\char`\&}
\def\PYZlt{\char`\<}
\def\PYZgt{\char`\>}
\def\PYZsh{\char`\#}
\def\PYZpc{\char`\%}
\def\PYZdl{\char`\$}
\def\PYZhy{\char`\-}
\def\PYZsq{\char`\'}
\def\PYZdq{\char`\"}
\def\PYZti{\char`\~}
% for compatibility with earlier versions
\def\PYZat{@}
\def\PYZlb{[}
\def\PYZrb{]}
\makeatother


    % For linebreaks inside Verbatim environment from package fancyvrb. 
    \makeatletter
        \newbox\Wrappedcontinuationbox 
        \newbox\Wrappedvisiblespacebox 
        \newcommand*\Wrappedvisiblespace {\textcolor{red}{\textvisiblespace}} 
        \newcommand*\Wrappedcontinuationsymbol {\textcolor{red}{\llap{\tiny$\m@th\hookrightarrow$}}} 
        \newcommand*\Wrappedcontinuationindent {3ex } 
        \newcommand*\Wrappedafterbreak {\kern\Wrappedcontinuationindent\copy\Wrappedcontinuationbox} 
        % Take advantage of the already applied Pygments mark-up to insert 
        % potential linebreaks for TeX processing. 
        %        {, <, #, %, $, ' and ": go to next line. 
        %        _, }, ^, &, >, - and ~: stay at end of broken line. 
        % Use of \textquotesingle for straight quote. 
        \newcommand*\Wrappedbreaksatspecials {% 
            \def\PYGZus{\discretionary{\char`\_}{\Wrappedafterbreak}{\char`\_}}% 
            \def\PYGZob{\discretionary{}{\Wrappedafterbreak\char`\{}{\char`\{}}% 
            \def\PYGZcb{\discretionary{\char`\}}{\Wrappedafterbreak}{\char`\}}}% 
            \def\PYGZca{\discretionary{\char`\^}{\Wrappedafterbreak}{\char`\^}}% 
            \def\PYGZam{\discretionary{\char`\&}{\Wrappedafterbreak}{\char`\&}}% 
            \def\PYGZlt{\discretionary{}{\Wrappedafterbreak\char`\<}{\char`\<}}% 
            \def\PYGZgt{\discretionary{\char`\>}{\Wrappedafterbreak}{\char`\>}}% 
            \def\PYGZsh{\discretionary{}{\Wrappedafterbreak\char`\#}{\char`\#}}% 
            \def\PYGZpc{\discretionary{}{\Wrappedafterbreak\char`\%}{\char`\%}}% 
            \def\PYGZdl{\discretionary{}{\Wrappedafterbreak\char`\$}{\char`\$}}% 
            \def\PYGZhy{\discretionary{\char`\-}{\Wrappedafterbreak}{\char`\-}}% 
            \def\PYGZsq{\discretionary{}{\Wrappedafterbreak\textquotesingle}{\textquotesingle}}% 
            \def\PYGZdq{\discretionary{}{\Wrappedafterbreak\char`\"}{\char`\"}}% 
            \def\PYGZti{\discretionary{\char`\~}{\Wrappedafterbreak}{\char`\~}}% 
        } 
        % Some characters . , ; ? ! / are not pygmentized. 
        % This macro makes them "active" and they will insert potential linebreaks 
        \newcommand*\Wrappedbreaksatpunct {% 
            \lccode`\~`\.\lowercase{\def~}{\discretionary{\hbox{\char`\.}}{\Wrappedafterbreak}{\hbox{\char`\.}}}% 
            \lccode`\~`\,\lowercase{\def~}{\discretionary{\hbox{\char`\,}}{\Wrappedafterbreak}{\hbox{\char`\,}}}% 
            \lccode`\~`\;\lowercase{\def~}{\discretionary{\hbox{\char`\;}}{\Wrappedafterbreak}{\hbox{\char`\;}}}% 
            \lccode`\~`\:\lowercase{\def~}{\discretionary{\hbox{\char`\:}}{\Wrappedafterbreak}{\hbox{\char`\:}}}% 
            \lccode`\~`\?\lowercase{\def~}{\discretionary{\hbox{\char`\?}}{\Wrappedafterbreak}{\hbox{\char`\?}}}% 
            \lccode`\~`\!\lowercase{\def~}{\discretionary{\hbox{\char`\!}}{\Wrappedafterbreak}{\hbox{\char`\!}}}% 
            \lccode`\~`\/\lowercase{\def~}{\discretionary{\hbox{\char`\/}}{\Wrappedafterbreak}{\hbox{\char`\/}}}% 
            \catcode`\.\active
            \catcode`\,\active 
            \catcode`\;\active
            \catcode`\:\active
            \catcode`\?\active
            \catcode`\!\active
            \catcode`\/\active 
            \lccode`\~`\~ 	
        }
    \makeatother

    \let\OriginalVerbatim=\Verbatim
    \makeatletter
    \renewcommand{\Verbatim}[1][1]{%
        %\parskip\z@skip
        \sbox\Wrappedcontinuationbox {\Wrappedcontinuationsymbol}%
        \sbox\Wrappedvisiblespacebox {\FV@SetupFont\Wrappedvisiblespace}%
        \def\FancyVerbFormatLine ##1{\hsize\linewidth
            \vtop{\raggedright\hyphenpenalty\z@\exhyphenpenalty\z@
                \doublehyphendemerits\z@\finalhyphendemerits\z@
                \strut ##1\strut}%
        }%
        % If the linebreak is at a space, the latter will be displayed as visible
        % space at end of first line, and a continuation symbol starts next line.
        % Stretch/shrink are however usually zero for typewriter font.
        \def\FV@Space {%
            \nobreak\hskip\z@ plus\fontdimen3\font minus\fontdimen4\font
            \discretionary{\copy\Wrappedvisiblespacebox}{\Wrappedafterbreak}
            {\kern\fontdimen2\font}%
        }%
        
        % Allow breaks at special characters using \PYG... macros.
        \Wrappedbreaksatspecials
        % Breaks at punctuation characters . , ; ? ! and / need catcode=\active 	
        \OriginalVerbatim[#1,codes*=\Wrappedbreaksatpunct]%
    }
    \makeatother

    % Exact colors from NB
    \definecolor{incolor}{HTML}{303F9F}
    \definecolor{outcolor}{HTML}{D84315}
    \definecolor{cellborder}{HTML}{CFCFCF}
    \definecolor{cellbackground}{HTML}{F7F7F7}
    
    % prompt
    \makeatletter
    \newcommand{\boxspacing}{\kern\kvtcb@left@rule\kern\kvtcb@boxsep}
    \makeatother
    \newcommand{\prompt}[4]{
        {\ttfamily\llap{{\color{#2}[#3]:\hspace{3pt}#4}}\vspace{-\baselineskip}}
    }
    

    
    % Prevent overflowing lines due to hard-to-break entities
    \sloppy 
    % Setup hyperref package
    \hypersetup{
      breaklinks=true,  % so long urls are correctly broken across lines
      colorlinks=true,
      urlcolor=urlcolor,
      linkcolor=linkcolor,
      citecolor=citecolor,
      }
    % Slightly bigger margins than the latex defaults
    
    \geometry{verbose,tmargin=1in,bmargin=1in,lmargin=1in,rmargin=1in}
    
    

\begin{document}
    
    \maketitle
    
    

    
    \hypertarget{introduction}{%
\section{Introduction}\label{introduction}}

These calculations assess the maximum retained height that can typically
be attained using the various Stepoc block types. In assessing the
maximum height the calculations consider:

\begin{itemize}
\tightlist
\item
  The maximum height to satisfy serviceability limit state accounting
  for the lateral yield of the wall
\item
  The maximum height to satisfy ultimate limit state based on a set of
  assumed soil parameters accounting for bending and shear
\end{itemize}

These calculations are for demonstration purposes only in establishing
the maximum retained height. The acutal maximum retained height will
depend on the actual soil and ground water condition, the magnitude of
the surcharge and the geometry of the retained soil.

The calculations follow the principles of BS EN 1990.

\hypertarget{ultimate-limit-state-soil-parameters-assumed}{%
\subsection{Ultimate limit state soil parameters
assumed}\label{ultimate-limit-state-soil-parameters-assumed}}

\begin{itemize}
\tightlist
\item
  The Worst credible bulk unit weight of the soil
  \(\gamma_b = 20 kN/m^3\)
\item
  Drain conditions for the soil with the design parameters \(C' = 0\)
  and \(\phi '= 32^{\circ}\)
\item
  An angle of wall friction \(\delta = 20^{\circ}\) with a horizontal
  retained surface
\item
  A nominal surcharge of \(5kN/m^2\)
\end{itemize}

Due to nominal yielding of the wall it will be assumed that full active
conditions exist in the soil block behind the wall with a Rankine
pressure distribution.

\hypertarget{analysis-of-the-wall-section}{%
\subsection{Analysis of the wall
section}\label{analysis-of-the-wall-section}}

The maximum bendding moment will be calculated using the derived
equation in the Structensor repot \emph{`A Proposal for Establishing the
Maximum Bending Capacity of Stepoc'}. The calculation of ULS shear
capacity is based on the recommendations within BS EN 1996-1-1 Annex J.
The SLS state in this report is assumed complied with by limiting the
span to effective depth ratio of the reinforced section to the limits
presented in BS EN 1996-1-1 Table 5.2.

\hypertarget{stepoc-units-considered}{%
\subsection{Stepoc units considered}\label{stepoc-units-considered}}

\begin{itemize}
\tightlist
\item
  Stepoc 200 with a wall thickness of 35mm and a 12mm diameter
  horizontal rebar central
\item
  Stepoc 256 with a wall thickness of 39mm and a horizontal bar 152mm
  from the compressive face
\item
  Stepoc 325 with a wall thickness of 39mm and a horizontal bar 186mm
  from the compressive face
\end{itemize}

In all cases, for the purposes of calculating the effective depth, a
vertical rebar of 16mm diameter is assumed.

\hypertarget{sls-calculations}{%
\section{SLS Calculations}\label{sls-calculations}}

Table 5.2 of BS EN 1996-1-1 places a limit of 18 on cantilevered
reinforced masonry elements subject to out of plane bending. This limit
leads to the following maximum heights for the stem of the retaining
wall from the top of the supporting foundation:

    \begin{tcolorbox}[breakable, size=fbox, boxrule=1pt, pad at break*=1mm,colback=cellbackground, colframe=cellborder]
\prompt{In}{incolor}{81}{\boxspacing}
\begin{Verbatim}[commandchars=\\\{\}]
\PY{n}{names} \PY{o}{=} \PY{p}{[}\PY{l+s+s1}{\PYZsq{}}\PY{l+s+s1}{Stepoc 200}\PY{l+s+s1}{\PYZsq{}}\PY{p}{,} \PY{l+s+s1}{\PYZsq{}}\PY{l+s+s1}{Stepoc 256}\PY{l+s+s1}{\PYZsq{}}\PY{p}{,} \PY{l+s+s1}{\PYZsq{}}\PY{l+s+s1}{Stepoc 325}\PY{l+s+s1}{\PYZsq{}}\PY{p}{]}
\PY{n}{stepocd} \PY{o}{=} \PY{p}{[}\PY{l+m+mi}{200}\PY{o}{/}\PY{l+m+mi}{2}\PY{p}{,} \PY{l+m+mi}{152}\PY{p}{,} \PY{l+m+mi}{186}\PY{p}{]} \PY{c+c1}{\PYZsh{} Values in mm}
\PY{n}{max\PYZus{}sls\PYZus{}height} \PY{o}{=} \PY{p}{[}\PY{p}{]}
\PY{k}{for} \PY{n}{i} \PY{o+ow}{in} \PY{n+nb}{range}\PY{p}{(}\PY{n+nb}{len}\PY{p}{(}\PY{n}{stepocd}\PY{p}{)}\PY{p}{)}\PY{p}{:}
    \PY{n}{h} \PY{o}{=} \PY{l+m+mi}{18} \PY{o}{*} \PY{n}{stepocd}\PY{p}{[}\PY{n}{i}\PY{p}{]}
    \PY{n}{max\PYZus{}sls\PYZus{}height}\PY{o}{.}\PY{n}{append}\PY{p}{(}\PY{n}{h}\PY{p}{)}
    \PY{n+nb}{print}\PY{p}{(}\PY{l+s+s1}{\PYZsq{}}\PY{l+s+s1}{Max SLS Stem height for }\PY{l+s+s1}{\PYZsq{}}\PY{p}{,} \PY{n}{names}\PY{p}{[}\PY{n}{i}\PY{p}{]}\PY{p}{,} \PY{l+s+s1}{\PYZsq{}}\PY{l+s+s1}{ = }\PY{l+s+si}{\PYZob{}:.2f\PYZcb{}}\PY{l+s+s1}{\PYZsq{}}\PY{o}{.}\PY{n}{format}\PY{p}{(}\PY{n}{h}\PY{p}{)}\PY{p}{)}
\end{Verbatim}
\end{tcolorbox}

    \begin{Verbatim}[commandchars=\\\{\}]
Max SLS Stem height for  Stepoc 200  = 1800.00
Max SLS Stem height for  Stepoc 256  = 2736.00
Max SLS Stem height for  Stepoc 325  = 3348.00
    \end{Verbatim}

    \hypertarget{uls-calculations}{%
\section{ULS Calculations}\label{uls-calculations}}

\hypertarget{bending}{%
\subsection{Bending}\label{bending}}

The maximum bending capacity of Stepoc in the Structensor report is
given by:

\begin{align}
M_{Rd} = b f_{md} \left(x_s (d - \frac{x_s}{2}) +  \alpha (x-x_s) (d - \frac{(x + x_s)}{2})\right)
\end{align}

Where to limit the depth of the compressive stress block the the
following is assumed:

\begin{align}
x=d(1-\sqrt{0.20})
\end{align}

The value of the modular ratio is assumed to be \(\alpha = 3\). The
following functions provides the maximum moment capacity of the
sections:

    \begin{tcolorbox}[breakable, size=fbox, boxrule=1pt, pad at break*=1mm,colback=cellbackground, colframe=cellborder]
\prompt{In}{incolor}{82}{\boxspacing}
\begin{Verbatim}[commandchars=\\\{\}]
\PY{k+kn}{import} \PY{n+nn}{pandas} \PY{k}{as} \PY{n+nn}{pd}
\PY{n}{pd}\PY{o}{.}\PY{n}{set\PYZus{}option}\PY{p}{(}\PY{l+s+s1}{\PYZsq{}}\PY{l+s+s1}{precision}\PY{l+s+s1}{\PYZsq{}}\PY{p}{,} \PY{l+m+mi}{3}\PY{p}{)}
\PY{k+kn}{import} \PY{n+nn}{math}
\PY{n}{xs} \PY{o}{=} \PY{p}{[}\PY{l+m+mi}{35}\PY{p}{,} \PY{l+m+mi}{39}\PY{p}{,} \PY{l+m+mi}{39}\PY{p}{]}
\PY{l+s+sd}{\PYZsq{}\PYZsq{}\PYZsq{} Set up data structure\PYZsq{}\PYZsq{}\PYZsq{}}
\PY{n}{moment\PYZus{}cap\PYZus{}data} \PY{o}{=} \PY{n}{pd}\PY{o}{.}\PY{n}{DataFrame}\PY{p}{(}\PY{p}{\PYZob{}}\PY{l+s+s1}{\PYZsq{}}\PY{l+s+s1}{d}\PY{l+s+s1}{\PYZsq{}}\PY{p}{:}\PY{n}{stepocd}\PY{p}{,} \PY{l+s+s1}{\PYZsq{}}\PY{l+s+s1}{xs}\PY{l+s+s1}{\PYZsq{}}\PY{p}{:}\PY{n}{xs}\PY{p}{\PYZcb{}}\PY{p}{)}

\PY{l+s+sd}{\PYZsq{}\PYZsq{}\PYZsq{} Calculate the value of x\PYZsq{}\PYZsq{}\PYZsq{}}
\PY{n}{moment\PYZus{}cap\PYZus{}data}\PY{p}{[}\PY{l+s+s1}{\PYZsq{}}\PY{l+s+s1}{x}\PY{l+s+s1}{\PYZsq{}}\PY{p}{]} \PY{o}{=} \PY{n}{moment\PYZus{}cap\PYZus{}data}\PY{p}{[}\PY{l+s+s1}{\PYZsq{}}\PY{l+s+s1}{d}\PY{l+s+s1}{\PYZsq{}}\PY{p}{]} \PY{o}{*} \PY{p}{(}\PY{l+m+mi}{1} \PY{o}{\PYZhy{}} \PY{n}{math}\PY{o}{.}\PY{n}{sqrt}\PY{p}{(}\PY{l+m+mf}{0.20}\PY{p}{)}\PY{p}{)}

\PY{l+s+sd}{\PYZsq{}\PYZsq{}\PYZsq{} Calculate the moment capacity\PYZsq{}\PYZsq{}\PYZsq{}}
\PY{n}{a} \PY{o}{=} \PY{l+m+mi}{3}
\PY{n}{b} \PY{o}{=} \PY{l+m+mi}{1000}
\PY{n}{fmd} \PY{o}{=} \PY{l+m+mf}{9.92}

\PY{n}{moment\PYZus{}cap\PYZus{}data}\PY{p}{[}\PY{l+s+s1}{\PYZsq{}}\PY{l+s+s1}{Mcap}\PY{l+s+s1}{\PYZsq{}}\PY{p}{]} \PY{o}{=} \PY{p}{(}\PY{p}{(}\PY{n}{moment\PYZus{}cap\PYZus{}data}\PY{p}{[}\PY{l+s+s1}{\PYZsq{}}\PY{l+s+s1}{xs}\PY{l+s+s1}{\PYZsq{}}\PY{p}{]}\PY{o}{*}\PY{p}{(}\PY{n}{moment\PYZus{}cap\PYZus{}data}\PY{p}{[}\PY{l+s+s1}{\PYZsq{}}\PY{l+s+s1}{d}\PY{l+s+s1}{\PYZsq{}}\PY{p}{]} \PY{o}{\PYZhy{}} \PY{n}{moment\PYZus{}cap\PYZus{}data}\PY{p}{[}\PY{l+s+s1}{\PYZsq{}}\PY{l+s+s1}{xs}\PY{l+s+s1}{\PYZsq{}}\PY{p}{]}\PY{o}{/}\PY{l+m+mi}{2}\PY{p}{)}
\PY{o}{+} \PY{n}{a} \PY{o}{*} \PY{p}{(}\PY{n}{moment\PYZus{}cap\PYZus{}data}\PY{p}{[}\PY{l+s+s1}{\PYZsq{}}\PY{l+s+s1}{x}\PY{l+s+s1}{\PYZsq{}}\PY{p}{]} \PY{o}{\PYZhy{}} \PY{n}{moment\PYZus{}cap\PYZus{}data}\PY{p}{[}\PY{l+s+s1}{\PYZsq{}}\PY{l+s+s1}{xs}\PY{l+s+s1}{\PYZsq{}}\PY{p}{]}\PY{p}{)} \PY{o}{*} \PY{p}{(}\PY{n}{moment\PYZus{}cap\PYZus{}data}\PY{p}{[}\PY{l+s+s1}{\PYZsq{}}\PY{l+s+s1}{d}\PY{l+s+s1}{\PYZsq{}}\PY{p}{]} \PY{o}{\PYZhy{}} \PY{p}{(}\PY{n}{moment\PYZus{}cap\PYZus{}data}\PY{p}{[}\PY{l+s+s1}{\PYZsq{}}\PY{l+s+s1}{x}\PY{l+s+s1}{\PYZsq{}}\PY{p}{]} \PY{o}{+} \PY{n}{moment\PYZus{}cap\PYZus{}data}\PY{p}{[}\PY{l+s+s1}{\PYZsq{}}\PY{l+s+s1}{xs}\PY{l+s+s1}{\PYZsq{}}\PY{p}{]}\PY{p}{)}\PY{o}{/}\PY{l+m+mi}{2}\PY{p}{)}\PY{p}{)}
\PY{o}{*}\PY{n}{b}\PY{o}{*}\PY{n}{fmd} \PY{o}{*}\PY{l+m+mi}{10}\PY{o}{*}\PY{o}{*}\PY{o}{\PYZhy{}}\PY{l+m+mi}{6}\PY{p}{)}

\PY{n}{display}\PY{p}{(}\PY{n}{moment\PYZus{}cap\PYZus{}data}\PY{p}{)}
\end{Verbatim}
\end{tcolorbox}

    
    \begin{Verbatim}[commandchars=\\\{\}]
       d  xs        x     Mcap
0  100.0  35   55.279   61.752
1  152.0  39   84.024  172.507
2  186.0  39  102.818  283.000
    \end{Verbatim}

    
    \hypertarget{shear}{%
\subsection{Shear}\label{shear}}

Within these calculations \(\gamma_M\) for masonry is taken to be 2.0.

Annex J in BS EN 1996-1-1 provides the following equation for the shear
strength of the section:

\begin{align}
f_{vd} = \frac{0.35 + 17.5 \rho}{\gamma_M}
\\\rho = \frac{A_s}{bd}
\end{align}

Providing that \(f_{vd}\) is not taken to be greater than
\(\frac{0.70}{\gamma_M}\).

\(f_{vd}\) may be enhanced by the following factor:

\begin{align}
\chi = (2.5 - 0.25 \frac{a_v}{d})
\end{align}

Where \(a_v\) is the shear span of the section (max bending moment / max
shear force in the section).

In order to use the above equation an area of steel needs to be
established. This can be achieved by using by provding equalibrium of
the section with the following equations for the force in the
compressive stress blocks:

\begin{align}
\text{In the masonry: }
F_m = \frac{b x_s f_{md}}{\gamma_{Mm}}\\
\text{In the concrete: }
F_c = \frac{b (x - x_s)f_{cd}}{\gamma_{Mc}}\\
\text{In total: } F_t = F_m + F_c
\end{align}

From this we can assume that \(A_s f_y / \gamma_M = F_t\) to give
\(A_s = F_t / (0.87 f_y)\).

These calculations can be added to the data set:

    \begin{tcolorbox}[breakable, size=fbox, boxrule=1pt, pad at break*=1mm,colback=cellbackground, colframe=cellborder]
\prompt{In}{incolor}{83}{\boxspacing}
\begin{Verbatim}[commandchars=\\\{\}]
\PY{n}{moment\PYZus{}cap\PYZus{}data}\PY{p}{[}\PY{l+s+s1}{\PYZsq{}}\PY{l+s+s1}{Fm}\PY{l+s+s1}{\PYZsq{}}\PY{p}{]} \PY{o}{=} \PY{n}{b} \PY{o}{*} \PY{n}{moment\PYZus{}cap\PYZus{}data}\PY{p}{[}\PY{l+s+s1}{\PYZsq{}}\PY{l+s+s1}{xs}\PY{l+s+s1}{\PYZsq{}}\PY{p}{]} \PY{o}{*} \PY{n}{fmd} \PY{o}{*} \PY{l+m+mi}{10}\PY{o}{*}\PY{o}{*}\PY{o}{\PYZhy{}}\PY{l+m+mi}{3} \PY{o}{/} \PY{l+m+mi}{2}
\PY{n}{moment\PYZus{}cap\PYZus{}data}\PY{p}{[}\PY{l+s+s1}{\PYZsq{}}\PY{l+s+s1}{Fc}\PY{l+s+s1}{\PYZsq{}}\PY{p}{]} \PY{o}{=} \PY{n}{b} \PY{o}{*} \PY{p}{(}\PY{n}{moment\PYZus{}cap\PYZus{}data}\PY{p}{[}\PY{l+s+s1}{\PYZsq{}}\PY{l+s+s1}{x}\PY{l+s+s1}{\PYZsq{}}\PY{p}{]} \PY{o}{\PYZhy{}} \PY{n}{moment\PYZus{}cap\PYZus{}data}\PY{p}{[}\PY{l+s+s1}{\PYZsq{}}\PY{l+s+s1}{xs}\PY{l+s+s1}{\PYZsq{}}\PY{p}{]}\PY{p}{)} \PY{o}{*} \PY{n}{a} \PY{o}{*} \PY{n}{fmd} \PY{o}{*} \PY{l+m+mi}{10}\PY{o}{*}\PY{o}{*}\PY{o}{\PYZhy{}}\PY{l+m+mi}{3} \PY{o}{/} \PY{l+m+mf}{1.5}
\PY{n}{moment\PYZus{}cap\PYZus{}data}\PY{p}{[}\PY{l+s+s1}{\PYZsq{}}\PY{l+s+s1}{Ft}\PY{l+s+s1}{\PYZsq{}}\PY{p}{]} \PY{o}{=} \PY{p}{(}\PY{n}{moment\PYZus{}cap\PYZus{}data}\PY{p}{[}\PY{l+s+s1}{\PYZsq{}}\PY{l+s+s1}{Fm}\PY{l+s+s1}{\PYZsq{}}\PY{p}{]}\PY{o}{+}\PY{n}{moment\PYZus{}cap\PYZus{}data}\PY{p}{[}\PY{l+s+s1}{\PYZsq{}}\PY{l+s+s1}{Fc}\PY{l+s+s1}{\PYZsq{}}\PY{p}{]}\PY{p}{)}
\PY{n}{moment\PYZus{}cap\PYZus{}data}\PY{p}{[}\PY{l+s+s1}{\PYZsq{}}\PY{l+s+s1}{As}\PY{l+s+s1}{\PYZsq{}}\PY{p}{]} \PY{o}{=} \PY{n}{moment\PYZus{}cap\PYZus{}data}\PY{p}{[}\PY{l+s+s1}{\PYZsq{}}\PY{l+s+s1}{Ft}\PY{l+s+s1}{\PYZsq{}}\PY{p}{]} \PY{o}{*} \PY{l+m+mi}{10}\PY{o}{*}\PY{o}{*}\PY{l+m+mi}{3} \PY{o}{/} \PY{p}{(}\PY{l+m+mf}{0.87} \PY{o}{*} \PY{l+m+mi}{500}\PY{p}{)}
\PY{n}{moment\PYZus{}cap\PYZus{}data}\PY{p}{[}\PY{l+s+s1}{\PYZsq{}}\PY{l+s+s1}{p}\PY{l+s+s1}{\PYZsq{}}\PY{p}{]} \PY{o}{=} \PY{n}{moment\PYZus{}cap\PYZus{}data}\PY{p}{[}\PY{l+s+s1}{\PYZsq{}}\PY{l+s+s1}{As}\PY{l+s+s1}{\PYZsq{}}\PY{p}{]} \PY{o}{/} \PY{p}{(}\PY{n}{b} \PY{o}{*} \PY{n}{moment\PYZus{}cap\PYZus{}data}\PY{p}{[}\PY{l+s+s1}{\PYZsq{}}\PY{l+s+s1}{d}\PY{l+s+s1}{\PYZsq{}}\PY{p}{]}\PY{p}{)}
\PY{n}{moment\PYZus{}cap\PYZus{}data}\PY{p}{[}\PY{l+s+s1}{\PYZsq{}}\PY{l+s+s1}{fvd}\PY{l+s+s1}{\PYZsq{}}\PY{p}{]} \PY{o}{=} \PY{p}{(}\PY{l+m+mf}{0.35} \PY{o}{+} \PY{l+m+mf}{17.5} \PY{o}{*} \PY{n}{moment\PYZus{}cap\PYZus{}data}\PY{o}{.}\PY{n}{p} \PY{o}{/} \PY{l+m+mi}{2}\PY{p}{)}
\PY{n}{moment\PYZus{}cap\PYZus{}data}\PY{p}{[}\PY{l+s+s1}{\PYZsq{}}\PY{l+s+s1}{fvdmax}\PY{l+s+s1}{\PYZsq{}}\PY{p}{]} \PY{o}{=} \PY{l+m+mf}{0.70} \PY{o}{/} \PY{l+m+mi}{2}
\PY{n}{moment\PYZus{}cap\PYZus{}data}\PY{p}{[}\PY{l+s+s1}{\PYZsq{}}\PY{l+s+s1}{Shear Cap}\PY{l+s+s1}{\PYZsq{}}\PY{p}{]} \PY{o}{=} \PY{n}{moment\PYZus{}cap\PYZus{}data}\PY{o}{.}\PY{n}{fvdmax} \PY{o}{*} \PY{n}{b} \PY{o}{*} \PY{n}{moment\PYZus{}cap\PYZus{}data}\PY{o}{.}\PY{n}{d} \PY{o}{*} \PY{l+m+mi}{10}\PY{o}{*}\PY{o}{*}\PY{o}{\PYZhy{}}\PY{l+m+mi}{3}
\PY{n}{display}\PY{p}{(}\PY{n}{moment\PYZus{}cap\PYZus{}data}\PY{p}{)}
\end{Verbatim}
\end{tcolorbox}

    
    \begin{Verbatim}[commandchars=\\\{\}]
       d  xs        x     Mcap      Fm        Fc        Ft        As      p  \textbackslash{}
0  100.0  35   55.279   61.752  173.60   402.328   575.928  1323.973  0.013   
1  152.0  39   84.024  172.507  193.44   893.267  1086.707  2498.177  0.016   
2  186.0  39  102.818  283.000  193.44  1266.155  1459.595  3355.390  0.018   

     fvd  fvdmax  Shear Cap  
0  0.466    0.35       35.0  
1  0.494    0.35       53.2  
2  0.508    0.35       65.1  
    \end{Verbatim}

    
    We can conclude from the above that it is possible to provide sufficient
reinforcement to balance the limited compressive stress block such that
the maximum limit of \(f_{vd}\) may be used in maximum value
calculations.

\hypertarget{force-effects-on-retaining-walls}{%
\section{Force effects on retaining
walls}\label{force-effects-on-retaining-walls}}

The active pressure coefficient for a granular soil is calculated from
the formula:

\begin{align}
k_a = \left(\frac{cosec(\psi) sin(\psi - \phi)}{\sqrt{sin(\psi - \delta)}
+ \sqrt{\frac{sin(\phi+\delta) sin(\phi - \beta)}{sin(\psi-\beta)}}}\right)^2
\end{align}

Where:

\begin{itemize}
\tightlist
\item
  \(\psi\) is the angle of the back of the wall from the horizontal
\item
  \(\phi\) is the internal angle of friction for the soil
\item
  \(\delta\) is the angle of frction to the back of the wall
\item
  \(\beta\) is the angle of the retained soil surface from the
  horizontal
\end{itemize}

\hypertarget{force-effects-on-the-retaining-wall}{%
\subsection{Force effects on the retaining
wall}\label{force-effects-on-the-retaining-wall}}

Pressure at a depth \(z\) below is given by: \begin{align}
p_a = k_a(\gamma_{fe} z \gamma_s + \gamma_{fq} q)
\end{align}

Where:

\begin{itemize}
\tightlist
\item
  \(\gamma_{fe}\) is the partial factor applied to earth unit force
  density
\item
  \(\gamma_{fq}\) is the partial factor applied to the surcharge
\item
  \(\gamma_s\) is the earth force density
\item
  \(q\) surcharge applied to the retained surface
\end{itemize}

From the above: \begin{align}
&V = \int_0^z k_a(\gamma_{fe} z \gamma_s + \gamma_{fq} q) cos(\delta) dz\\
&V = k_a(\frac{\gamma_{fe} \gamma_{s} z^2}{2} + \gamma_{fq}qz)cos(\delta)
\end{align}

V = shear force at depth z perpendicular to the back of the wall.

And moments \(M\): \begin{align}
&M = \int_0^zk_a \left(\frac{\gamma_{fe} \gamma_{s} z^2}{2} + \gamma_{fq}qz \right)cos(\delta)dz\\
&M = k_a\left(\frac{\gamma_{fe} \gamma_{s} z^3}{6} + \frac{\gamma_{fq}qz^2}{2} \right)cos(\delta)
\end{align}

The shear applied over the plane of the back of the \(F_v\) wall (or
virtual back) is: \begin{align}
F_v = V sin(\delta)
\end{align}

These equations are encoded in the methods below.

    \begin{tcolorbox}[breakable, size=fbox, boxrule=1pt, pad at break*=1mm,colback=cellbackground, colframe=cellborder]
\prompt{In}{incolor}{84}{\boxspacing}
\begin{Verbatim}[commandchars=\\\{\}]
\PY{k+kn}{import} \PY{n+nn}{math}

\PY{k}{def} \PY{n+nf}{ka}\PY{p}{(}\PY{n}{psi}\PY{p}{,} \PY{n}{phi}\PY{p}{,} \PY{n}{delta}\PY{p}{,} \PY{n}{beta}\PY{p}{)}\PY{p}{:}
    \PY{c+c1}{\PYZsh{} convert to radians}
    \PY{n}{psi} \PY{o}{=} \PY{n}{math}\PY{o}{.}\PY{n}{radians}\PY{p}{(}\PY{n}{psi}\PY{p}{)}
    \PY{n}{phi} \PY{o}{=} \PY{n}{math}\PY{o}{.}\PY{n}{radians}\PY{p}{(}\PY{n}{phi}\PY{p}{)}
    \PY{n}{delta} \PY{o}{=} \PY{n}{math}\PY{o}{.}\PY{n}{radians}\PY{p}{(}\PY{n}{delta}\PY{p}{)}
    \PY{n}{beta} \PY{o}{=} \PY{n}{math}\PY{o}{.}\PY{n}{radians}\PY{p}{(}\PY{n}{beta}\PY{p}{)}
    \PY{n}{a} \PY{o}{=} \PY{n}{math}\PY{o}{.}\PY{n}{sin}\PY{p}{(}\PY{n}{psi} \PY{o}{\PYZhy{}} \PY{n}{phi}\PY{p}{)} \PY{o}{/} \PY{n}{math}\PY{o}{.}\PY{n}{sin}\PY{p}{(}\PY{n}{psi}\PY{p}{)}
    \PY{n}{b} \PY{o}{=} \PY{n}{math}\PY{o}{.}\PY{n}{sqrt}\PY{p}{(}\PY{n}{math}\PY{o}{.}\PY{n}{sin}\PY{p}{(}\PY{n}{psi} \PY{o}{\PYZhy{}} \PY{n}{delta}\PY{p}{)}\PY{p}{)}
    \PY{n}{c} \PY{o}{=} \PY{n}{math}\PY{o}{.}\PY{n}{sqrt}\PY{p}{(}\PY{n}{math}\PY{o}{.}\PY{n}{sin}\PY{p}{(}\PY{n}{phi} \PY{o}{+} \PY{n}{delta}\PY{p}{)} \PY{o}{*} \PY{n}{math}\PY{o}{.}\PY{n}{sin}\PY{p}{(}\PY{n}{phi} \PY{o}{\PYZhy{}} \PY{n}{beta}\PY{p}{)}
                  \PY{o}{/} \PY{n}{math}\PY{o}{.}\PY{n}{sin}\PY{p}{(}\PY{n}{psi} \PY{o}{\PYZhy{}} \PY{n}{beta}\PY{p}{)}\PY{p}{)}
    \PY{n}{ka} \PY{o}{=} \PY{p}{(}\PY{n}{a} \PY{o}{/} \PY{p}{(}\PY{n}{b} \PY{o}{+} \PY{n}{c}\PY{p}{)}\PY{p}{)}\PY{o}{*}\PY{o}{*}\PY{l+m+mi}{2}
    \PY{k}{return} \PY{n}{ka}
\PY{k}{def} \PY{n+nf}{p}\PY{p}{(}\PY{n}{ka}\PY{p}{,} \PY{n}{z}\PY{p}{,} \PY{n}{gs}\PY{p}{,} \PY{n}{gfe}\PY{p}{,} \PY{n}{gfq}\PY{p}{,} \PY{n}{q}\PY{p}{)}\PY{p}{:}
    \PY{k}{return} \PY{n}{ka}\PY{o}{*}\PY{p}{(}\PY{n}{gfe} \PY{o}{*} \PY{n}{gs} \PY{o}{*} \PY{n}{z} \PY{o}{+} \PY{n}{gfq} \PY{o}{*} \PY{n}{q}\PY{p}{)}
\PY{k}{def} \PY{n+nf}{V}\PY{p}{(}\PY{n}{ka}\PY{p}{,} \PY{n}{delta}\PY{p}{,} \PY{n}{z}\PY{p}{,} \PY{n}{gs}\PY{p}{,} \PY{n}{gfe}\PY{p}{,} \PY{n}{gfq}\PY{p}{,} \PY{n}{q}\PY{p}{)}\PY{p}{:}
    \PY{n}{delta} \PY{o}{=} \PY{n}{math}\PY{o}{.}\PY{n}{radians}\PY{p}{(}\PY{n}{delta}\PY{p}{)}
    \PY{n}{earth} \PY{o}{=} \PY{n}{gfe} \PY{o}{*} \PY{n}{gs} \PY{o}{*} \PY{n}{z}\PY{o}{*}\PY{o}{*}\PY{l+m+mi}{2}\PY{o}{/}\PY{l+m+mi}{2}
    \PY{n}{surcharge} \PY{o}{=} \PY{n}{gfq} \PY{o}{*} \PY{n}{q} \PY{o}{*} \PY{n}{z}
    \PY{k}{return} \PY{n}{math}\PY{o}{.}\PY{n}{cos}\PY{p}{(}\PY{n}{delta}\PY{p}{)} \PY{o}{*} \PY{n}{ka} \PY{o}{*} \PY{p}{(}\PY{n}{earth} \PY{o}{+} \PY{n}{surcharge}\PY{p}{)}

\PY{k}{def} \PY{n+nf}{M}\PY{p}{(}\PY{n}{ka}\PY{p}{,} \PY{n}{delta}\PY{p}{,} \PY{n}{z}\PY{p}{,} \PY{n}{gs}\PY{p}{,} \PY{n}{gfe}\PY{p}{,} \PY{n}{gfq}\PY{p}{,} \PY{n}{q}\PY{p}{)}\PY{p}{:}
    \PY{n}{delta} \PY{o}{=} \PY{n}{math}\PY{o}{.}\PY{n}{radians}\PY{p}{(}\PY{n}{delta}\PY{p}{)}
    \PY{n}{earth} \PY{o}{=} \PY{n}{gfe} \PY{o}{*} \PY{n}{gs} \PY{o}{*} \PY{n}{z}\PY{o}{*}\PY{o}{*}\PY{l+m+mi}{3} \PY{o}{/} \PY{l+m+mi}{6}
    \PY{n}{surcharge} \PY{o}{=} \PY{n}{gfq} \PY{o}{*} \PY{n}{q} \PY{o}{*} \PY{n}{z}\PY{o}{*}\PY{o}{*}\PY{l+m+mi}{2} \PY{o}{/} \PY{l+m+mi}{2}
    \PY{k}{return} \PY{n}{ka} \PY{o}{*} \PY{n}{math}\PY{o}{.}\PY{n}{cos}\PY{p}{(}\PY{n}{delta}\PY{p}{)}\PY{o}{*}\PY{p}{(}\PY{n}{earth} \PY{o}{+} \PY{n}{surcharge}\PY{p}{)}
\end{Verbatim}
\end{tcolorbox}

    Establish the input paramneters to use the above functions:

    \begin{tcolorbox}[breakable, size=fbox, boxrule=1pt, pad at break*=1mm,colback=cellbackground, colframe=cellborder]
\prompt{In}{incolor}{85}{\boxspacing}
\begin{Verbatim}[commandchars=\\\{\}]
\PY{l+s+sd}{\PYZsq{}\PYZsq{}\PYZsq{}Input data from ultimate limit state soil parameters\PYZsq{}\PYZsq{}\PYZsq{}}
\PY{n}{soil\PYZus{}weight} \PY{o}{=} \PY{l+m+mi}{20}
\PY{n}{wall\PYZus{}friction} \PY{o}{=} \PY{l+m+mi}{20}
\PY{n}{soil\PYZus{}friction} \PY{o}{=} \PY{l+m+mi}{32}
\PY{n}{surcharge} \PY{o}{=} \PY{l+m+mi}{5}
\PY{n}{gfe} \PY{o}{=} \PY{l+m+mf}{1.35}
\PY{n}{gfq} \PY{o}{=} \PY{l+m+mf}{1.50}
\PY{n}{k\PYZus{}active} \PY{o}{=} \PY{n}{ka}\PY{p}{(}\PY{l+m+mi}{90}\PY{p}{,} \PY{n}{soil\PYZus{}friction}\PY{p}{,} \PY{n}{wall\PYZus{}friction}\PY{p}{,} \PY{l+m+mi}{0}\PY{p}{)}
\end{Verbatim}
\end{tcolorbox}

    \hypertarget{establish-the-value-of-shear-enhancement-factor-chi}{%
\subsection{\texorpdfstring{Establish the value of shear enhancement
factor
(\(\chi\))}{Establish the value of shear enhancement factor (\textbackslash chi)}}\label{establish-the-value-of-shear-enhancement-factor-chi}}

Plot the shear strength enhancement factor as a function of depth and
layer over that the maximum stem heights calculated for SLS above.

    \begin{tcolorbox}[breakable, size=fbox, boxrule=1pt, pad at break*=1mm,colback=cellbackground, colframe=cellborder]
\prompt{In}{incolor}{86}{\boxspacing}
\begin{Verbatim}[commandchars=\\\{\}]
\PY{k+kn}{import} \PY{n+nn}{matplotlib}\PY{n+nn}{.}\PY{n+nn}{pyplot} \PY{k}{as} \PY{n+nn}{plt}
\PY{k+kn}{import} \PY{n+nn}{numpy} \PY{k}{as} \PY{n+nn}{np}
\PY{k+kn}{import} \PY{n+nn}{warnings}
\PY{n}{warnings}\PY{o}{.}\PY{n}{filterwarnings}\PY{p}{(}\PY{l+s+s1}{\PYZsq{}}\PY{l+s+s1}{ignore}\PY{l+s+s1}{\PYZsq{}}\PY{p}{)}
\PY{n}{warnings}\PY{o}{.}\PY{n}{simplefilter}\PY{p}{(}\PY{l+s+s1}{\PYZsq{}}\PY{l+s+s1}{ignore}\PY{l+s+s1}{\PYZsq{}}\PY{p}{)}
\PY{l+s+sd}{\PYZsq{}\PYZsq{}\PYZsq{}Plot the shear strength enhancement factors over depth\PYZsq{}\PYZsq{}\PYZsq{}}
\PY{n}{plt}\PY{o}{.}\PY{n}{figure}\PY{p}{(}\PY{n}{figsize}\PY{o}{=}\PY{p}{(}\PY{l+m+mi}{8}\PY{p}{,}\PY{l+m+mi}{8}\PY{p}{)}\PY{p}{)}
\PY{n}{x} \PY{o}{=} \PY{n}{np}\PY{o}{.}\PY{n}{arange}\PY{p}{(}\PY{l+m+mi}{0}\PY{p}{,}\PY{l+m+mi}{4}\PY{p}{,}\PY{l+m+mf}{0.10}\PY{p}{)}
\PY{n}{va} \PY{o}{=} \PY{n}{V}\PY{p}{(}\PY{n}{k\PYZus{}active}\PY{p}{,} \PY{n}{wall\PYZus{}friction}\PY{p}{,} \PY{n}{x}\PY{p}{,} \PY{n}{soil\PYZus{}weight}\PY{p}{,} \PY{n}{gfe}\PY{p}{,} \PY{n}{gfq}\PY{p}{,} \PY{n}{surcharge}\PY{p}{)}
\PY{n}{ma} \PY{o}{=} \PY{n}{M}\PY{p}{(}\PY{n}{k\PYZus{}active}\PY{p}{,} \PY{n}{wall\PYZus{}friction}\PY{p}{,} \PY{n}{x}\PY{p}{,} \PY{n}{soil\PYZus{}weight}\PY{p}{,} \PY{n}{gfe}\PY{p}{,} \PY{n}{gfq}\PY{p}{,} \PY{n}{surcharge}\PY{p}{)}
\PY{n}{av} \PY{o}{=} \PY{n}{ma}\PY{o}{/}\PY{n}{va}
\PY{n}{plt}\PY{o}{.}\PY{n}{plot}\PY{p}{(}\PY{n}{x}\PY{p}{,} \PY{p}{(}\PY{l+m+mf}{2.50} \PY{o}{\PYZhy{}} \PY{l+m+mf}{0.25} \PY{o}{*} \PY{p}{(}\PY{n}{av} \PY{o}{*} \PY{l+m+mi}{1000} \PY{o}{/} \PY{n}{stepocd}\PY{p}{[}\PY{l+m+mi}{0}\PY{p}{]}\PY{p}{)}\PY{p}{)}\PY{p}{,} \PY{n}{label}\PY{o}{=}\PY{l+s+s1}{\PYZsq{}}\PY{l+s+s1}{200 Stepoc}\PY{l+s+s1}{\PYZsq{}}\PY{p}{)}
\PY{n}{plt}\PY{o}{.}\PY{n}{plot}\PY{p}{(}\PY{n}{x}\PY{p}{,} \PY{p}{(}\PY{l+m+mf}{2.50} \PY{o}{\PYZhy{}} \PY{l+m+mf}{0.25} \PY{o}{*} \PY{p}{(}\PY{n}{av} \PY{o}{*} \PY{l+m+mi}{1000} \PY{o}{/} \PY{n}{stepocd}\PY{p}{[}\PY{l+m+mi}{1}\PY{p}{]}\PY{p}{)}\PY{p}{)}\PY{p}{,} \PY{n}{label}\PY{o}{=}\PY{l+s+s1}{\PYZsq{}}\PY{l+s+s1}{256 Stepoc}\PY{l+s+s1}{\PYZsq{}}\PY{p}{)}
\PY{n}{plt}\PY{o}{.}\PY{n}{plot}\PY{p}{(}\PY{n}{x}\PY{p}{,} \PY{p}{(}\PY{l+m+mf}{2.50} \PY{o}{\PYZhy{}} \PY{l+m+mf}{0.25} \PY{o}{*} \PY{p}{(}\PY{n}{av} \PY{o}{*} \PY{l+m+mi}{1000} \PY{o}{/} \PY{n}{stepocd}\PY{p}{[}\PY{l+m+mi}{2}\PY{p}{]}\PY{p}{)}\PY{p}{)}\PY{p}{,} \PY{n}{label}\PY{o}{=}\PY{l+s+s1}{\PYZsq{}}\PY{l+s+s1}{325 Stepoc}\PY{l+s+s1}{\PYZsq{}}\PY{p}{)}
\PY{n}{plt}\PY{o}{.}\PY{n}{axvline}\PY{p}{(}\PY{n}{max\PYZus{}sls\PYZus{}height}\PY{p}{[}\PY{l+m+mi}{0}\PY{p}{]}\PY{o}{/}\PY{l+m+mi}{1000}\PY{p}{,} \PY{n}{label}\PY{o}{=}\PY{l+s+s1}{\PYZsq{}}\PY{l+s+s1}{200 Stepoc SLS height}\PY{l+s+s1}{\PYZsq{}}\PY{p}{,} \PY{n}{color}\PY{o}{=}\PY{l+s+s1}{\PYZsq{}}\PY{l+s+s1}{b}\PY{l+s+s1}{\PYZsq{}}\PY{p}{)}
\PY{n}{plt}\PY{o}{.}\PY{n}{axvline}\PY{p}{(}\PY{n}{max\PYZus{}sls\PYZus{}height}\PY{p}{[}\PY{l+m+mi}{1}\PY{p}{]}\PY{o}{/}\PY{l+m+mi}{1000}\PY{p}{,} \PY{n}{label}\PY{o}{=}\PY{l+s+s1}{\PYZsq{}}\PY{l+s+s1}{256 Stepoc SLS height}\PY{l+s+s1}{\PYZsq{}}\PY{p}{,} \PY{n}{color}\PY{o}{=}\PY{l+s+s1}{\PYZsq{}}\PY{l+s+s1}{y}\PY{l+s+s1}{\PYZsq{}}\PY{p}{)}
\PY{n}{plt}\PY{o}{.}\PY{n}{axvline}\PY{p}{(}\PY{n}{max\PYZus{}sls\PYZus{}height}\PY{p}{[}\PY{l+m+mi}{2}\PY{p}{]}\PY{o}{/}\PY{l+m+mi}{1000}\PY{p}{,} \PY{n}{label}\PY{o}{=}\PY{l+s+s1}{\PYZsq{}}\PY{l+s+s1}{325 Stepoc SLS height}\PY{l+s+s1}{\PYZsq{}}\PY{p}{,} \PY{n}{color}\PY{o}{=}\PY{l+s+s1}{\PYZsq{}}\PY{l+s+s1}{g}\PY{l+s+s1}{\PYZsq{}}\PY{p}{)}
\PY{n}{plt}\PY{o}{.}\PY{n}{xlabel}\PY{p}{(}\PY{l+s+s1}{\PYZsq{}}\PY{l+s+s1}{Depth (m)}\PY{l+s+s1}{\PYZsq{}}\PY{p}{)}
\PY{n}{plt}\PY{o}{.}\PY{n}{ylabel}\PY{p}{(}\PY{l+s+s1}{\PYZsq{}}\PY{l+s+s1}{Shear span (m)}\PY{l+s+s1}{\PYZsq{}}\PY{p}{)}
\PY{n}{plt}\PY{o}{.}\PY{n}{title}\PY{p}{(}\PY{l+s+s1}{\PYZsq{}}\PY{l+s+s1}{Shear span as a fucntion of depth}\PY{l+s+s1}{\PYZsq{}}\PY{p}{)}
\PY{n}{plt}\PY{o}{.}\PY{n}{legend}\PY{p}{(}\PY{p}{)}
\PY{n}{plt}\PY{o}{.}\PY{n}{grid}\PY{p}{(}\PY{p}{)}
\PY{n}{plt}\PY{o}{.}\PY{n}{show}\PY{p}{(}\PY{p}{)}
\end{Verbatim}
\end{tcolorbox}

    \begin{center}
    \adjustimage{max size={0.9\linewidth}{0.9\paperheight}}{Stepoc maximum retained height_files/Stepoc maximum retained height_11_0.png}
    \end{center}
    { \hspace*{\fill} \\}
    
    The above plots demonstrate that in the vincinty of the maximum SLS
height the shear strength ehancement factor is less than 1. It is
therefore ignoed in shear strength plots below.

    \hypertarget{retaining-wall-foce-effects-plot-with-stepoc-capacities-layered-over}{%
\subsection{Retaining wall foce effects plot with Stepoc capacities
layered
over}\label{retaining-wall-foce-effects-plot-with-stepoc-capacities-layered-over}}

    \begin{tcolorbox}[breakable, size=fbox, boxrule=1pt, pad at break*=1mm,colback=cellbackground, colframe=cellborder]
\prompt{In}{incolor}{87}{\boxspacing}
\begin{Verbatim}[commandchars=\\\{\}]
\PY{l+s+sd}{\PYZsq{}\PYZsq{}\PYZsq{}Graphical plot of force effects\PYZsq{}\PYZsq{}\PYZsq{}}
\PY{n}{plt}\PY{o}{.}\PY{n}{figure}\PY{p}{(}\PY{n}{figsize}\PY{o}{=}\PY{p}{(}\PY{l+m+mi}{10}\PY{p}{,}\PY{l+m+mi}{10}\PY{p}{)}\PY{p}{)}
\PY{n}{x} \PY{o}{=} \PY{n}{np}\PY{o}{.}\PY{n}{arange}\PY{p}{(}\PY{l+m+mi}{0}\PY{p}{,}\PY{l+m+mi}{4}\PY{p}{,}\PY{l+m+mf}{0.20}\PY{p}{)}
\PY{n}{va} \PY{o}{=} \PY{n}{V}\PY{p}{(}\PY{n}{k\PYZus{}active}\PY{p}{,} \PY{n}{wall\PYZus{}friction}\PY{p}{,} \PY{n}{x}\PY{p}{,} \PY{n}{soil\PYZus{}weight}\PY{p}{,} \PY{n}{gfe}\PY{p}{,} \PY{n}{gfq}\PY{p}{,} \PY{n}{surcharge}\PY{p}{)}
\PY{n}{ma} \PY{o}{=} \PY{n}{M}\PY{p}{(}\PY{n}{k\PYZus{}active}\PY{p}{,} \PY{n}{wall\PYZus{}friction}\PY{p}{,} \PY{n}{x}\PY{p}{,} \PY{n}{soil\PYZus{}weight}\PY{p}{,} \PY{n}{gfe}\PY{p}{,} \PY{n}{gfq}\PY{p}{,} \PY{n}{surcharge}\PY{p}{)}
\PY{n}{plt}\PY{o}{.}\PY{n}{plot}\PY{p}{(}\PY{n}{x}\PY{p}{,} \PY{n}{va}\PY{p}{,} \PY{n}{label}\PY{o}{=}\PY{l+s+s1}{\PYZsq{}}\PY{l+s+s1}{Shear (kN)}\PY{l+s+s1}{\PYZsq{}}\PY{p}{)}
\PY{n}{plt}\PY{o}{.}\PY{n}{fill\PYZus{}between}\PY{p}{(}\PY{n}{x}\PY{p}{,} \PY{n}{va}\PY{p}{)}
\PY{n}{plt}\PY{o}{.}\PY{n}{plot}\PY{p}{(}\PY{n}{x}\PY{p}{,} \PY{n}{ma}\PY{p}{,} \PY{n}{label}\PY{o}{=}\PY{l+s+s1}{\PYZsq{}}\PY{l+s+s1}{Moment (kNm)}\PY{l+s+s1}{\PYZsq{}}\PY{p}{)}
\PY{n}{plt}\PY{o}{.}\PY{n}{fill\PYZus{}between}\PY{p}{(}\PY{n}{x}\PY{p}{,} \PY{n}{ma}\PY{p}{)}
\PY{n}{plt}\PY{o}{.}\PY{n}{axhline}\PY{p}{(}\PY{n}{moment\PYZus{}cap\PYZus{}data}\PY{o}{.}\PY{n}{iloc}\PY{p}{[}\PY{l+m+mi}{0}\PY{p}{,}\PY{l+m+mi}{11}\PY{p}{]}\PY{p}{,} \PY{n}{label}\PY{o}{=}\PY{l+s+s1}{\PYZsq{}}\PY{l+s+s1}{200 Stepoc V}\PY{l+s+s1}{\PYZsq{}}\PY{p}{,} \PY{n}{color}\PY{o}{=}\PY{l+s+s1}{\PYZsq{}}\PY{l+s+s1}{r}\PY{l+s+s1}{\PYZsq{}}\PY{p}{)}
\PY{n}{plt}\PY{o}{.}\PY{n}{axhline}\PY{p}{(}\PY{n}{moment\PYZus{}cap\PYZus{}data}\PY{o}{.}\PY{n}{iloc}\PY{p}{[}\PY{l+m+mi}{1}\PY{p}{,}\PY{l+m+mi}{11}\PY{p}{]}\PY{p}{,} \PY{n}{label}\PY{o}{=}\PY{l+s+s1}{\PYZsq{}}\PY{l+s+s1}{256 Stepoc V}\PY{l+s+s1}{\PYZsq{}}\PY{p}{,} \PY{n}{color}\PY{o}{=}\PY{l+s+s1}{\PYZsq{}}\PY{l+s+s1}{g}\PY{l+s+s1}{\PYZsq{}}\PY{p}{)}
\PY{n}{plt}\PY{o}{.}\PY{n}{axhline}\PY{p}{(}\PY{n}{moment\PYZus{}cap\PYZus{}data}\PY{o}{.}\PY{n}{iloc}\PY{p}{[}\PY{l+m+mi}{2}\PY{p}{,}\PY{l+m+mi}{11}\PY{p}{]}\PY{p}{,} \PY{n}{label}\PY{o}{=}\PY{l+s+s1}{\PYZsq{}}\PY{l+s+s1}{325 Stepoc V}\PY{l+s+s1}{\PYZsq{}}\PY{p}{,} \PY{n}{color}\PY{o}{=}\PY{l+s+s1}{\PYZsq{}}\PY{l+s+s1}{y}\PY{l+s+s1}{\PYZsq{}}\PY{p}{)}
\PY{n}{plt}\PY{o}{.}\PY{n}{axvline}\PY{p}{(}\PY{n}{max\PYZus{}sls\PYZus{}height}\PY{p}{[}\PY{l+m+mi}{0}\PY{p}{]}\PY{o}{/}\PY{l+m+mi}{1000}\PY{p}{,} \PY{n}{label}\PY{o}{=}\PY{l+s+s1}{\PYZsq{}}\PY{l+s+s1}{200 Stepoc SLS}\PY{l+s+s1}{\PYZsq{}}\PY{p}{,} \PY{n}{color}\PY{o}{=}\PY{l+s+s1}{\PYZsq{}}\PY{l+s+s1}{r}\PY{l+s+s1}{\PYZsq{}}\PY{p}{)}
\PY{n}{plt}\PY{o}{.}\PY{n}{axvline}\PY{p}{(}\PY{n}{max\PYZus{}sls\PYZus{}height}\PY{p}{[}\PY{l+m+mi}{1}\PY{p}{]}\PY{o}{/}\PY{l+m+mi}{1000}\PY{p}{,} \PY{n}{label} \PY{o}{=} \PY{l+s+s1}{\PYZsq{}}\PY{l+s+s1}{256 Stepoc SLS}\PY{l+s+s1}{\PYZsq{}}\PY{p}{,} \PY{n}{color}\PY{o}{=}\PY{l+s+s1}{\PYZsq{}}\PY{l+s+s1}{g}\PY{l+s+s1}{\PYZsq{}}\PY{p}{)}
\PY{n}{plt}\PY{o}{.}\PY{n}{axvline}\PY{p}{(}\PY{n}{max\PYZus{}sls\PYZus{}height}\PY{p}{[}\PY{l+m+mi}{2}\PY{p}{]}\PY{o}{/}\PY{l+m+mi}{1000}\PY{p}{,} \PY{n}{label} \PY{o}{=} \PY{l+s+s1}{\PYZsq{}}\PY{l+s+s1}{325 Stepoc SLS}\PY{l+s+s1}{\PYZsq{}}\PY{p}{,} \PY{n}{color}\PY{o}{=}\PY{l+s+s1}{\PYZsq{}}\PY{l+s+s1}{y}\PY{l+s+s1}{\PYZsq{}}\PY{p}{)}
\PY{n}{plt}\PY{o}{.}\PY{n}{axhline}\PY{p}{(}\PY{n}{moment\PYZus{}cap\PYZus{}data}\PY{o}{.}\PY{n}{iloc}\PY{p}{[}\PY{l+m+mi}{0}\PY{p}{,}\PY{l+m+mi}{3}\PY{p}{]}\PY{p}{,} \PY{n}{label}\PY{o}{=}\PY{l+s+s1}{\PYZsq{}}\PY{l+s+s1}{200 Stepoc M}\PY{l+s+s1}{\PYZsq{}}\PY{p}{,} \PY{n}{color}\PY{o}{=}\PY{l+s+s1}{\PYZsq{}}\PY{l+s+s1}{r}\PY{l+s+s1}{\PYZsq{}}\PY{p}{,} \PY{n}{linestyle}\PY{o}{=}\PY{l+s+s1}{\PYZsq{}}\PY{l+s+s1}{\PYZhy{}\PYZhy{}}\PY{l+s+s1}{\PYZsq{}}\PY{p}{)}
\PY{n}{plt}\PY{o}{.}\PY{n}{axhline}\PY{p}{(}\PY{n}{moment\PYZus{}cap\PYZus{}data}\PY{o}{.}\PY{n}{iloc}\PY{p}{[}\PY{l+m+mi}{1}\PY{p}{,}\PY{l+m+mi}{3}\PY{p}{]}\PY{p}{,} \PY{n}{label}\PY{o}{=}\PY{l+s+s1}{\PYZsq{}}\PY{l+s+s1}{256 Stepoc M}\PY{l+s+s1}{\PYZsq{}}\PY{p}{,} \PY{n}{color}\PY{o}{=}\PY{l+s+s1}{\PYZsq{}}\PY{l+s+s1}{g}\PY{l+s+s1}{\PYZsq{}}\PY{p}{,} \PY{n}{linestyle}\PY{o}{=}\PY{l+s+s1}{\PYZsq{}}\PY{l+s+s1}{\PYZhy{}\PYZhy{}}\PY{l+s+s1}{\PYZsq{}}\PY{p}{)}
\PY{n}{plt}\PY{o}{.}\PY{n}{axhline}\PY{p}{(}\PY{n}{moment\PYZus{}cap\PYZus{}data}\PY{o}{.}\PY{n}{iloc}\PY{p}{[}\PY{l+m+mi}{2}\PY{p}{,}\PY{l+m+mi}{3}\PY{p}{]}\PY{p}{,} \PY{n}{label}\PY{o}{=}\PY{l+s+s1}{\PYZsq{}}\PY{l+s+s1}{325 Stepoc M}\PY{l+s+s1}{\PYZsq{}}\PY{p}{,} \PY{n}{color}\PY{o}{=}\PY{l+s+s1}{\PYZsq{}}\PY{l+s+s1}{y}\PY{l+s+s1}{\PYZsq{}}\PY{p}{,} \PY{n}{linestyle}\PY{o}{=}\PY{l+s+s1}{\PYZsq{}}\PY{l+s+s1}{\PYZhy{}\PYZhy{}}\PY{l+s+s1}{\PYZsq{}}\PY{p}{)}
\PY{n}{plt}\PY{o}{.}\PY{n}{grid}\PY{p}{(}\PY{p}{)}
\PY{n}{plt}\PY{o}{.}\PY{n}{title}\PY{p}{(}\PY{l+s+s1}{\PYZsq{}}\PY{l+s+s1}{Force effects on the wall with wall capacities overlaid}\PY{l+s+s1}{\PYZsq{}}\PY{p}{)}
\PY{n}{plt}\PY{o}{.}\PY{n}{legend}\PY{p}{(}\PY{p}{)}
\PY{n}{plt}\PY{o}{.}\PY{n}{xlabel}\PY{p}{(}\PY{l+s+s1}{\PYZsq{}}\PY{l+s+s1}{Depth (m)}\PY{l+s+s1}{\PYZsq{}}\PY{p}{)}
\PY{n}{plt}\PY{o}{.}\PY{n}{ylabel}\PY{p}{(}\PY{l+s+s1}{\PYZsq{}}\PY{l+s+s1}{Force effect}\PY{l+s+s1}{\PYZsq{}}\PY{p}{)}
\PY{n}{plt}\PY{o}{.}\PY{n}{show}\PY{p}{(}\PY{p}{)}
\end{Verbatim}
\end{tcolorbox}

    \begin{center}
    \adjustimage{max size={0.9\linewidth}{0.9\paperheight}}{Stepoc maximum retained height_files/Stepoc maximum retained height_14_0.png}
    \end{center}
    { \hspace*{\fill} \\}
    
    The above capacity plots demonstrate that in all cases the limiting SLS
conditon from BS EN 1996-1-1 (Table 5.1) governs the maximum retaining
wall stem height under the conditions assumed.

    \hypertarget{conclusions}{%
\section{Conclusions}\label{conclusions}}

The calculations within this report, for the soil conditions assumed,
demonstrate that the deemed to satisfy serviceability limit state
governs the maximum retained height for the stem.

Under the conditions within this report, being typical of drained
retaining wall conditions with nominal surcharge the following are the
maximum retained heights:

\begin{itemize}
\tightlist
\item
  200 Stepoc max height = 1.80 m
\item
  256 Stepoc max hieght = 2.70 m
\item
  325 Stepoc max height = 3.30 m
\end{itemize}


    % Add a bibliography block to the postdoc
    
    
    
\end{document}
